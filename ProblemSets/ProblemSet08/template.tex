\documentclass{report}

%%%%%%%%%%%%%%%%%%%%%%%%%%%%%%%%%
% PACKAGE IMPORTS
%%%%%%%%%%%%%%%%%%%%%%%%%%%%%%%%%


\usepackage[tmargin=2cm,rmargin=1in,lmargin=1in,margin=0.85in,bmargin=2cm,footskip=.2in]{geometry}
\usepackage{amsmath,amsfonts,amsthm,amssymb,mathtools}
\usepackage[varbb]{newpxmath}
\usepackage{xfrac}
\usepackage[makeroom]{cancel}
\usepackage{mathtools}
\usepackage{bookmark}
\usepackage{enumitem}
\usepackage{hyperref,theoremref}
\hypersetup{
	pdftitle={Assignment},
	colorlinks=true, linkcolor=doc!90,
	bookmarksnumbered=true,
	bookmarksopen=true
}
\usepackage[most,many,breakable]{tcolorbox}
\usepackage{xcolor}
\usepackage{varwidth}
\usepackage{varwidth}
\usepackage{etoolbox}
%\usepackage{authblk}
\usepackage{nameref}
\usepackage{multicol,array}
\usepackage{tikz-cd}
\usepackage[ruled,vlined,linesnumbered]{algorithm2e}
\usepackage{comment} % enables the use of multi-line comments (\ifx \fi) 
\usepackage{import}
\usepackage{xifthen}
\usepackage{pdfpages}
\usepackage{transparent}

\newcommand\mycommfont[1]{\footnotesize\ttfamily\textcolor{blue}{#1}}
\SetCommentSty{mycommfont}
\newcommand{\incfig}[1]{%
    \def\svgwidth{\columnwidth}
    \import{./figures/}{#1.pdf_tex}
}

\usepackage{tikzsymbols}
\renewcommand\qedsymbol{$\Laughey$}


%\usepackage{import}
%\usepackage{xifthen}
%\usepackage{pdfpages}
%\usepackage{transparent}


%%%%%%%%%%%%%%%%%%%%%%%%%%%%%%
% SELF MADE COLORS
%%%%%%%%%%%%%%%%%%%%%%%%%%%%%%



\definecolor{myg}{RGB}{56, 140, 70}
\definecolor{myb}{RGB}{45, 111, 177}
\definecolor{myr}{RGB}{199, 68, 64}
\definecolor{mytheorembg}{HTML}{F2F2F9}
\definecolor{mytheoremfr}{HTML}{00007B}
\definecolor{mylenmabg}{HTML}{FFFAF8}
\definecolor{mylenmafr}{HTML}{983b0f}
\definecolor{mypropbg}{HTML}{f2fbfc}
\definecolor{mypropfr}{HTML}{191971}
\definecolor{myexamplebg}{HTML}{F2FBF8}
\definecolor{myexamplefr}{HTML}{88D6D1}
\definecolor{myexampleti}{HTML}{2A7F7F}
\definecolor{mydefinitbg}{HTML}{E5E5FF}
\definecolor{mydefinitfr}{HTML}{3F3FA3}
\definecolor{notesgreen}{RGB}{0,162,0}
\definecolor{myp}{RGB}{197, 92, 212}
\definecolor{mygr}{HTML}{2C3338}
\definecolor{myred}{RGB}{127,0,0}
\definecolor{myyellow}{RGB}{169,121,69}
\definecolor{myexercisebg}{HTML}{F2FBF8}
\definecolor{myexercisefg}{HTML}{88D6D1}


%%%%%%%%%%%%%%%%%%%%%%%%%%%%
% TCOLORBOX SETUPS
%%%%%%%%%%%%%%%%%%%%%%%%%%%%

\setlength{\parindent}{1cm}
%================================
% THEOREM BOX
%================================

\tcbuselibrary{theorems,skins,hooks}
\newtcbtheorem[number within=section]{Theorem}{Theorem}
{%
	enhanced,
	breakable,
	colback = mytheorembg,
	frame hidden,
	boxrule = 0sp,
	borderline west = {2pt}{0pt}{mytheoremfr},
	sharp corners,
	detach title,
	before upper = \tcbtitle\par\smallskip,
	coltitle = mytheoremfr,
	fonttitle = \bfseries\sffamily,
	description font = \mdseries,
	separator sign none,
	segmentation style={solid, mytheoremfr},
}
{th}

\tcbuselibrary{theorems,skins,hooks}
\newtcbtheorem[number within=chapter]{theorem}{Theorem}
{%
	enhanced,
	breakable,
	colback = mytheorembg,
	frame hidden,
	boxrule = 0sp,
	borderline west = {2pt}{0pt}{mytheoremfr},
	sharp corners,
	detach title,
	before upper = \tcbtitle\par\smallskip,
	coltitle = mytheoremfr,
	fonttitle = \bfseries\sffamily,
	description font = \mdseries,
	separator sign none,
	segmentation style={solid, mytheoremfr},
}
{th}


\tcbuselibrary{theorems,skins,hooks}
\newtcolorbox{Theoremcon}
{%
	enhanced
	,breakable
	,colback = mytheorembg
	,frame hidden
	,boxrule = 0sp
	,borderline west = {2pt}{0pt}{mytheoremfr}
	,sharp corners
	,description font = \mdseries
	,separator sign none
}

%================================
% Corollery
%================================
\tcbuselibrary{theorems,skins,hooks}
\newtcbtheorem[number within=section]{Corollary}{Corollary}
{%
	enhanced
	,breakable
	,colback = myp!10
	,frame hidden
	,boxrule = 0sp
	,borderline west = {2pt}{0pt}{myp!85!black}
	,sharp corners
	,detach title
	,before upper = \tcbtitle\par\smallskip
	,coltitle = myp!85!black
	,fonttitle = \bfseries\sffamily
	,description font = \mdseries
	,separator sign none
	,segmentation style={solid, myp!85!black}
}
{th}
\tcbuselibrary{theorems,skins,hooks}
\newtcbtheorem[number within=chapter]{corollary}{Corollary}
{%
	enhanced
	,breakable
	,colback = myp!10
	,frame hidden
	,boxrule = 0sp
	,borderline west = {2pt}{0pt}{myp!85!black}
	,sharp corners
	,detach title
	,before upper = \tcbtitle\par\smallskip
	,coltitle = myp!85!black
	,fonttitle = \bfseries\sffamily
	,description font = \mdseries
	,separator sign none
	,segmentation style={solid, myp!85!black}
}
{th}


%================================
% LENMA
%================================

\tcbuselibrary{theorems,skins,hooks}
\newtcbtheorem[number within=section]{Lenma}{Lenma}
{%
	enhanced,
	breakable,
	colback = mylenmabg,
	frame hidden,
	boxrule = 0sp,
	borderline west = {2pt}{0pt}{mylenmafr},
	sharp corners,
	detach title,
	before upper = \tcbtitle\par\smallskip,
	coltitle = mylenmafr,
	fonttitle = \bfseries\sffamily,
	description font = \mdseries,
	separator sign none,
	segmentation style={solid, mylenmafr},
}
{th}

\tcbuselibrary{theorems,skins,hooks}
\newtcbtheorem[number within=chapter]{lenma}{Lenma}
{%
	enhanced,
	breakable,
	colback = mylenmabg,
	frame hidden,
	boxrule = 0sp,
	borderline west = {2pt}{0pt}{mylenmafr},
	sharp corners,
	detach title,
	before upper = \tcbtitle\par\smallskip,
	coltitle = mylenmafr,
	fonttitle = \bfseries\sffamily,
	description font = \mdseries,
	separator sign none,
	segmentation style={solid, mylenmafr},
}
{th}


%================================
% PROPOSITION
%================================

\tcbuselibrary{theorems,skins,hooks}
\newtcbtheorem[number within=section]{Prop}{Proposition}
{%
	enhanced,
	breakable,
	colback = mypropbg,
	frame hidden,
	boxrule = 0sp,
	borderline west = {2pt}{0pt}{mypropfr},
	sharp corners,
	detach title,
	before upper = \tcbtitle\par\smallskip,
	coltitle = mypropfr,
	fonttitle = \bfseries\sffamily,
	description font = \mdseries,
	separator sign none,
	segmentation style={solid, mypropfr},
}
{th}

\tcbuselibrary{theorems,skins,hooks}
\newtcbtheorem[number within=chapter]{prop}{Proposition}
{%
	enhanced,
	breakable,
	colback = mypropbg,
	frame hidden,
	boxrule = 0sp,
	borderline west = {2pt}{0pt}{mypropfr},
	sharp corners,
	detach title,
	before upper = \tcbtitle\par\smallskip,
	coltitle = mypropfr,
	fonttitle = \bfseries\sffamily,
	description font = \mdseries,
	separator sign none,
	segmentation style={solid, mypropfr},
}
{th}


%================================
% CLAIM
%================================

\tcbuselibrary{theorems,skins,hooks}
\newtcbtheorem[number within=section]{claim}{Claim}
{%
	enhanced
	,breakable
	,colback = myg!10
	,frame hidden
	,boxrule = 0sp
	,borderline west = {2pt}{0pt}{myg}
	,sharp corners
	,detach title
	,before upper = \tcbtitle\par\smallskip
	,coltitle = myg!85!black
	,fonttitle = \bfseries\sffamily
	,description font = \mdseries
	,separator sign none
	,segmentation style={solid, myg!85!black}
}
{th}



%================================
% Exercise
%================================

\tcbuselibrary{theorems,skins,hooks}
\newtcbtheorem[number within=section]{Exercise}{Exercise}
{%
	enhanced,
	breakable,
	colback = myexercisebg,
	frame hidden,
	boxrule = 0sp,
	borderline west = {2pt}{0pt}{myexercisefg},
	sharp corners,
	detach title,
	before upper = \tcbtitle\par\smallskip,
	coltitle = myexercisefg,
	fonttitle = \bfseries\sffamily,
	description font = \mdseries,
	separator sign none,
	segmentation style={solid, myexercisefg},
}
{th}

\tcbuselibrary{theorems,skins,hooks}
\newtcbtheorem[number within=chapter]{exercise}{Exercise}
{%
	enhanced,
	breakable,
	colback = myexercisebg,
	frame hidden,
	boxrule = 0sp,
	borderline west = {2pt}{0pt}{myexercisefg},
	sharp corners,
	detach title,
	before upper = \tcbtitle\par\smallskip,
	coltitle = myexercisefg,
	fonttitle = \bfseries\sffamily,
	description font = \mdseries,
	separator sign none,
	segmentation style={solid, myexercisefg},
}
{th}

%================================
% EXAMPLE BOX
%================================

\newtcbtheorem[number within=section]{Example}{Example}
{%
	colback = myexamplebg
	,breakable
	,colframe = myexamplefr
	,coltitle = myexampleti
	,boxrule = 1pt
	,sharp corners
	,detach title
	,before upper=\tcbtitle\par\smallskip
	,fonttitle = \bfseries
	,description font = \mdseries
	,separator sign none
	,description delimiters parenthesis
}
{ex}

\newtcbtheorem[number within=chapter]{example}{Example}
{%
	colback = myexamplebg
	,breakable
	,colframe = myexamplefr
	,coltitle = myexampleti
	,boxrule = 1pt
	,sharp corners
	,detach title
	,before upper=\tcbtitle\par\smallskip
	,fonttitle = \bfseries
	,description font = \mdseries
	,separator sign none
	,description delimiters parenthesis
}
{ex}

%================================
% DEFINITION BOX
%================================

\newtcbtheorem[number within=section]{Definition}{Definition}{enhanced,
	before skip=2mm,after skip=2mm, colback=red!5,colframe=red!80!black,boxrule=0.5mm,
	attach boxed title to top left={xshift=1cm,yshift*=1mm-\tcboxedtitleheight}, varwidth boxed title*=-3cm,
	boxed title style={frame code={
					\path[fill=tcbcolback]
					([yshift=-1mm,xshift=-1mm]frame.north west)
					arc[start angle=0,end angle=180,radius=1mm]
					([yshift=-1mm,xshift=1mm]frame.north east)
					arc[start angle=180,end angle=0,radius=1mm];
					\path[left color=tcbcolback!60!black,right color=tcbcolback!60!black,
						middle color=tcbcolback!80!black]
					([xshift=-2mm]frame.north west) -- ([xshift=2mm]frame.north east)
					[rounded corners=1mm]-- ([xshift=1mm,yshift=-1mm]frame.north east)
					-- (frame.south east) -- (frame.south west)
					-- ([xshift=-1mm,yshift=-1mm]frame.north west)
					[sharp corners]-- cycle;
				},interior engine=empty,
		},
	fonttitle=\bfseries,
	title={#2},#1}{def}
\newtcbtheorem[number within=chapter]{definition}{Definition}{enhanced,
	before skip=2mm,after skip=2mm, colback=red!5,colframe=red!80!black,boxrule=0.5mm,
	attach boxed title to top left={xshift=1cm,yshift*=1mm-\tcboxedtitleheight}, varwidth boxed title*=-3cm,
	boxed title style={frame code={
					\path[fill=tcbcolback]
					([yshift=-1mm,xshift=-1mm]frame.north west)
					arc[start angle=0,end angle=180,radius=1mm]
					([yshift=-1mm,xshift=1mm]frame.north east)
					arc[start angle=180,end angle=0,radius=1mm];
					\path[left color=tcbcolback!60!black,right color=tcbcolback!60!black,
						middle color=tcbcolback!80!black]
					([xshift=-2mm]frame.north west) -- ([xshift=2mm]frame.north east)
					[rounded corners=1mm]-- ([xshift=1mm,yshift=-1mm]frame.north east)
					-- (frame.south east) -- (frame.south west)
					-- ([xshift=-1mm,yshift=-1mm]frame.north west)
					[sharp corners]-- cycle;
				},interior engine=empty,
		},
	fonttitle=\bfseries,
	title={#2},#1}{def}



%================================
% Solution BOX
%================================

\makeatletter
\newtcbtheorem{question}{Question}{enhanced,
	breakable,
	colback=white,
	colframe=myb!80!black,
	attach boxed title to top left={yshift*=-\tcboxedtitleheight},
	fonttitle=\bfseries,
	title={#2},
	boxed title size=title,
	boxed title style={%
			sharp corners,
			rounded corners=northwest,
			colback=tcbcolframe,
			boxrule=0pt,
		},
	underlay boxed title={%
			\path[fill=tcbcolframe] (title.south west)--(title.south east)
			to[out=0, in=180] ([xshift=5mm]title.east)--
			(title.center-|frame.east)
			[rounded corners=\kvtcb@arc] |-
			(frame.north) -| cycle;
		},
	#1
}{def}
\makeatother

%================================
% SOLUTION BOX
%================================

\makeatletter
\newtcolorbox{solution}{enhanced,
	breakable,
	colback=white,
	colframe=myg!80!black,
	attach boxed title to top left={yshift*=-\tcboxedtitleheight},
	title=Solution,
	boxed title size=title,
	boxed title style={%
			sharp corners,
			rounded corners=northwest,
			colback=tcbcolframe,
			boxrule=0pt,
		},
	underlay boxed title={%
			\path[fill=tcbcolframe] (title.south west)--(title.south east)
			to[out=0, in=180] ([xshift=5mm]title.east)--
			(title.center-|frame.east)
			[rounded corners=\kvtcb@arc] |-
			(frame.north) -| cycle;
		},
}
\makeatother

%================================
% Question BOX
%================================

\makeatletter
\newtcbtheorem{qstion}{Question}{enhanced,
	breakable,
	colback=white,
	colframe=mygr,
	attach boxed title to top left={yshift*=-\tcboxedtitleheight},
	fonttitle=\bfseries,
	title={#2},
	boxed title size=title,
	boxed title style={%
			sharp corners,
			rounded corners=northwest,
			colback=tcbcolframe,
			boxrule=0pt,
		},
	underlay boxed title={%
			\path[fill=tcbcolframe] (title.south west)--(title.south east)
			to[out=0, in=180] ([xshift=5mm]title.east)--
			(title.center-|frame.east)
			[rounded corners=\kvtcb@arc] |-
			(frame.north) -| cycle;
		},
	#1
}{def}
\makeatother

\newtcbtheorem[number within=chapter]{wconc}{Wrong Concept}{
	breakable,
	enhanced,
	colback=white,
	colframe=myr,
	arc=0pt,
	outer arc=0pt,
	fonttitle=\bfseries\sffamily\large,
	colbacktitle=myr,
	attach boxed title to top left={},
	boxed title style={
			enhanced,
			skin=enhancedfirst jigsaw,
			arc=3pt,
			bottom=0pt,
			interior style={fill=myr}
		},
	#1
}{def}



%================================
% NOTE BOX
%================================

\usetikzlibrary{arrows,calc,shadows.blur}
\tcbuselibrary{skins}
\newtcolorbox{note}[1][]{%
	enhanced jigsaw,
	colback=gray!20!white,%
	colframe=gray!80!black,
	size=small,
	boxrule=1pt,
	title=\textbf{Note:-},
	halign title=flush center,
	coltitle=black,
	breakable,
	drop shadow=black!50!white,
	attach boxed title to top left={xshift=1cm,yshift=-\tcboxedtitleheight/2,yshifttext=-\tcboxedtitleheight/2},
	minipage boxed title=1.5cm,
	boxed title style={%
			colback=white,
			size=fbox,
			boxrule=1pt,
			boxsep=2pt,
			underlay={%
					\coordinate (dotA) at ($(interior.west) + (-0.5pt,0)$);
					\coordinate (dotB) at ($(interior.east) + (0.5pt,0)$);
					\begin{scope}
						\clip (interior.north west) rectangle ([xshift=3ex]interior.east);
						\filldraw [white, blur shadow={shadow opacity=60, shadow yshift=-.75ex}, rounded corners=2pt] (interior.north west) rectangle (interior.south east);
					\end{scope}
					\begin{scope}[gray!80!black]
						\fill (dotA) circle (2pt);
						\fill (dotB) circle (2pt);
					\end{scope}
				},
		},
	#1,
}

%%%%%%%%%%%%%%%%%%%%%%%%%%%%%%
% SELF MADE COMMANDS
%%%%%%%%%%%%%%%%%%%%%%%%%%%%%%


\newcommand{\thm}[2]{\begin{Theorem}{#1}{}#2\end{Theorem}}
\newcommand{\cor}[2]{\begin{Corollary}{#1}{}#2\end{Corollary}}
\newcommand{\mlenma}[2]{\begin{Lenma}{#1}{}#2\end{Lenma}}
\newcommand{\mprop}[2]{\begin{Prop}{#1}{}#2\end{Prop}}
\newcommand{\clm}[3]{\begin{claim}{#1}{#2}#3\end{claim}}
\newcommand{\wc}[2]{\begin{wconc}{#1}{}\setlength{\parindent}{1cm}#2\end{wconc}}
\newcommand{\thmcon}[1]{\begin{Theoremcon}{#1}\end{Theoremcon}}
\newcommand{\ex}[2]{\begin{Example}{#1}{}#2\end{Example}}
\newcommand{\dfn}[2]{\begin{Definition}[colbacktitle=red!75!black]{#1}{}#2\end{Definition}}
\newcommand{\dfnc}[2]{\begin{definition}[colbacktitle=red!75!black]{#1}{}#2\end{definition}}
\newcommand{\qs}[2]{\begin{question}{#1}{}#2\end{question}}
\newcommand{\pf}[2]{\begin{myproof}[#1]#2\end{myproof}}
\newcommand{\nt}[1]{\begin{note}#1\end{note}}

\newcommand*\circled[1]{\tikz[baseline=(char.base)]{
		\node[shape=circle,draw,inner sep=1pt] (char) {#1};}}
\newcommand\getcurrentref[1]{%
	\ifnumequal{\value{#1}}{0}
	{??}
	{\the\value{#1}}%
}
\newcommand{\getCurrentSectionNumber}{\getcurrentref{section}}
\newenvironment{myproof}[1][\proofname]{%
	\proof[\bfseries #1: ]%
}{\endproof}

\newcommand{\mclm}[2]{\begin{myclaim}[#1]#2\end{myclaim}}
\newenvironment{myclaim}[1][\claimname]{\proof[\bfseries #1: ]}{}

\newcounter{mylabelcounter}

\makeatletter
\newcommand{\setword}[2]{%
	\phantomsection
	#1\def\@currentlabel{\unexpanded{#1}}\label{#2}%
}
\makeatother




\tikzset{
	symbol/.style={
			draw=none,
			every to/.append style={
					edge node={node [sloped, allow upside down, auto=false]{$#1$}}}
		}
}


% deliminators
\DeclarePairedDelimiter{\abs}{\lvert}{\rvert}
\DeclarePairedDelimiter{\norm}{\lVert}{\rVert}

\DeclarePairedDelimiter{\ceil}{\lceil}{\rceil}
\DeclarePairedDelimiter{\floor}{\lfloor}{\rfloor}
\DeclarePairedDelimiter{\round}{\lfloor}{\rceil}

\newsavebox\diffdbox
\newcommand{\slantedromand}{{\mathpalette\makesl{d}}}
\newcommand{\makesl}[2]{%
\begingroup
\sbox{\diffdbox}{$\mathsurround=0pt#1\mathrm{#2}$}%
\pdfsave
\pdfsetmatrix{1 0 0.2 1}%
\rlap{\usebox{\diffdbox}}%
\pdfrestore
\hskip\wd\diffdbox
\endgroup
}
\newcommand{\dd}[1][]{\ensuremath{\mathop{}\!\ifstrempty{#1}{%
\slantedromand\@ifnextchar^{\hspace{0.2ex}}{\hspace{0.1ex}}}%
{\slantedromand\hspace{0.2ex}^{#1}}}}
\ProvideDocumentCommand\dv{o m g}{%
  \ensuremath{%
    \IfValueTF{#3}{%
      \IfNoValueTF{#1}{%
        \frac{\dd #2}{\dd #3}%
      }{%
        \frac{\dd^{#1} #2}{\dd #3^{#1}}%
      }%
    }{%
      \IfNoValueTF{#1}{%
        \frac{\dd}{\dd #2}%
      }{%
        \frac{\dd^{#1}}{\dd #2^{#1}}%
      }%
    }%
  }%
}
\providecommand*{\pdv}[3][]{\frac{\partial^{#1}#2}{\partial#3^{#1}}}
%  - others
\DeclareMathOperator{\Lap}{\mathcal{L}}
\DeclareMathOperator{\Var}{Var} % varience
\DeclareMathOperator{\Cov}{Cov} % covarience
\DeclareMathOperator{\E}{E} % expected

% Since the amsthm package isn't loaded

% I prefer the slanted \leq
\let\oldleq\leq % save them in case they're every wanted
\let\oldgeq\geq
\renewcommand{\leq}{\leqslant}
\renewcommand{\geq}{\geqslant}

% % redefine matrix env to allow for alignment, use r as default
% \renewcommand*\env@matrix[1][r]{\hskip -\arraycolsep
%     \let\@ifnextchar\new@ifnextchar
%     \array{*\c@MaxMatrixCols #1}}


%\usepackage{framed}
%\usepackage{titletoc}
%\usepackage{etoolbox}
%\usepackage{lmodern}


%\patchcmd{\tableofcontents}{\contentsname}{\sffamily\contentsname}{}{}

%\renewenvironment{leftbar}
%{\def\FrameCommand{\hspace{6em}%
%		{\color{myyellow}\vrule width 2pt depth 6pt}\hspace{1em}}%
%	\MakeFramed{\parshape 1 0cm \dimexpr\textwidth-6em\relax\FrameRestore}\vskip2pt%
%}
%{\endMakeFramed}

%\titlecontents{chapter}
%[0em]{\vspace*{2\baselineskip}}
%{\parbox{4.5em}{%
%		\hfill\Huge\sffamily\bfseries\color{myred}\thecontentspage}%
%	\vspace*{-2.3\baselineskip}\leftbar\textsc{\small\chaptername~\thecontentslabel}\\\sffamily}
%{}{\endleftbar}
%\titlecontents{section}
%[8.4em]
%{\sffamily\contentslabel{3em}}{}{}
%{\hspace{0.5em}\nobreak\itshape\color{myred}\contentspage}
%\titlecontents{subsection}
%[8.4em]
%{\sffamily\contentslabel{3em}}{}{}  
%{\hspace{0.5em}\nobreak\itshape\color{myred}\contentspage}



%%%%%%%%%%%%%%%%%%%%%%%%%%%%%%%%%%%%%%%%%%%
% TABLE OF CONTENTS
%%%%%%%%%%%%%%%%%%%%%%%%%%%%%%%%%%%%%%%%%%%

\usepackage{tikz}
\definecolor{doc}{RGB}{0,60,110}
\usepackage{titletoc}
\contentsmargin{0cm}
\titlecontents{chapter}[3.7pc]
{\addvspace{30pt}%
	\begin{tikzpicture}[remember picture, overlay]%
		\draw[fill=doc!60,draw=doc!60] (-7,-.1) rectangle (-0.9,.5);%
		\pgftext[left,x=-3.5cm,y=0.2cm]{\color{white}\Large\sc\bfseries Chapter\ \thecontentslabel};%
	\end{tikzpicture}\color{doc!60}\large\sc\bfseries}%
{}
{}
{\;\titlerule\;\large\sc\bfseries Page \thecontentspage
	\begin{tikzpicture}[remember picture, overlay]
		\draw[fill=doc!60,draw=doc!60] (2pt,0) rectangle (4,0.1pt);
	\end{tikzpicture}}%
\titlecontents{section}[3.7pc]
{\addvspace{2pt}}
{\contentslabel[\thecontentslabel]{2pc}}
{}
{\hfill\small \thecontentspage}
[]
\titlecontents*{subsection}[3.7pc]
{\addvspace{-1pt}\small}
{}
{}
{\ --- \small\thecontentspage}
[ \textbullet\ ][]

\makeatletter
\renewcommand{\tableofcontents}{%
	\chapter*{%
	  \vspace*{-20\p@}%
	  \begin{tikzpicture}[remember picture, overlay]%
		  \pgftext[right,x=15cm,y=0.2cm]{\color{doc!60}\Huge\sc\bfseries \contentsname};%
		  \draw[fill=doc!60,draw=doc!60] (13,-.75) rectangle (20,1);%
		  \clip (13,-.75) rectangle (20,1);
		  \pgftext[right,x=15cm,y=0.2cm]{\color{white}\Huge\sc\bfseries \contentsname};%
	  \end{tikzpicture}}%
	\@starttoc{toc}}
\makeatother


%From M275 "Topology" at SJSU
\newcommand{\id}{\mathrm{id}}
\newcommand{\taking}[1]{\xrightarrow{#1}}
\newcommand{\inv}{^{-1}}

%From M170 "Introduction to Graph Theory" at SJSU
\DeclareMathOperator{\diam}{diam}
\DeclareMathOperator{\ord}{ord}
\newcommand{\defeq}{\overset{\mathrm{def}}{=}}

%From the USAMO .tex files
\newcommand{\ts}{\textsuperscript}
\newcommand{\dg}{^\circ}
\newcommand{\ii}{\item}

% % From Math 55 and Math 145 at Harvard
% \newenvironment{subproof}[1][Proof]{%
% \begin{proof}[#1] \renewcommand{\qedsymbol}{$\blacksquare$}}%
% {\end{proof}}

\newcommand{\liff}{\leftrightarrow}
\newcommand{\lthen}{\rightarrow}
\newcommand{\opname}{\operatorname}
\newcommand{\surjto}{\twoheadrightarrow}
\newcommand{\injto}{\hookrightarrow}
\newcommand{\On}{\mathrm{On}} % ordinals
\DeclareMathOperator{\img}{im} % Image
\DeclareMathOperator{\Img}{Im} % Image
\DeclareMathOperator{\coker}{coker} % Cokernel
\DeclareMathOperator{\Coker}{Coker} % Cokernel
\DeclareMathOperator{\Ker}{Ker} % Kernel
\DeclareMathOperator{\rank}{rank}
\DeclareMathOperator{\Spec}{Spec} % spectrum
\DeclareMathOperator{\Tr}{Tr} % trace
\DeclareMathOperator{\pr}{pr} % projection
\DeclareMathOperator{\ext}{ext} % extension
\DeclareMathOperator{\pred}{pred} % predecessor
\DeclareMathOperator{\dom}{dom} % domain
\DeclareMathOperator{\ran}{ran} % range
\DeclareMathOperator{\Hom}{Hom} % homomorphism
\DeclareMathOperator{\Mor}{Mor} % morphisms
\DeclareMathOperator{\End}{End} % endomorphism

\newcommand{\eps}{\epsilon}
\newcommand{\veps}{\varepsilon}
\newcommand{\ol}{\overline}
\newcommand{\ul}{\underline}
\newcommand{\wt}{\widetilde}
\newcommand{\wh}{\widehat}
\newcommand{\vocab}[1]{\textbf{\color{blue} #1}}
\providecommand{\half}{\frac{1}{2}}
\newcommand{\dang}{\measuredangle} %% Directed angle
\newcommand{\ray}[1]{\overrightarrow{#1}}
\newcommand{\seg}[1]{\overline{#1}}
\newcommand{\arc}[1]{\wideparen{#1}}
\DeclareMathOperator{\cis}{cis}
\DeclareMathOperator*{\lcm}{lcm}
\DeclareMathOperator*{\argmin}{arg min}
\DeclareMathOperator*{\argmax}{arg max}
\newcommand{\cycsum}{\sum_{\mathrm{cyc}}}
\newcommand{\symsum}{\sum_{\mathrm{sym}}}
\newcommand{\cycprod}{\prod_{\mathrm{cyc}}}
\newcommand{\symprod}{\prod_{\mathrm{sym}}}
\newcommand{\Qed}{\begin{flushright}\qed\end{flushright}}
\newcommand{\parinn}{\setlength{\parindent}{1cm}}
\newcommand{\parinf}{\setlength{\parindent}{0cm}}
% \newcommand{\norm}{\|\cdot\|}
\newcommand{\inorm}{\norm_{\infty}}
\newcommand{\opensets}{\{V_{\alpha}\}_{\alpha\in I}}
\newcommand{\oset}{V_{\alpha}}
\newcommand{\opset}[1]{V_{\alpha_{#1}}}
\newcommand{\lub}{\text{lub}}
\newcommand{\del}[2]{\frac{\partial #1}{\partial #2}}
\newcommand{\Del}[3]{\frac{\partial^{#1} #2}{\partial^{#1} #3}}
\newcommand{\deld}[2]{\dfrac{\partial #1}{\partial #2}}
\newcommand{\Deld}[3]{\dfrac{\partial^{#1} #2}{\partial^{#1} #3}}
\newcommand{\lm}{\lambda}
\newcommand{\uin}{\mathbin{\rotatebox[origin=c]{90}{$\in$}}}
\newcommand{\usubset}{\mathbin{\rotatebox[origin=c]{90}{$\subset$}}}
\newcommand{\lt}{\left}
\newcommand{\rt}{\right}
\newcommand{\bs}[1]{\boldsymbol{#1}}
\newcommand{\exs}{\exists}
\newcommand{\st}{\strut}
\newcommand{\dps}[1]{\displaystyle{#1}}

\newcommand{\sol}{\setlength{\parindent}{0cm}\textbf{\textit{Solution:}}\setlength{\parindent}{1cm} }
\newcommand{\solve}[1]{\setlength{\parindent}{0cm}\textbf{\textit{Solution: }}\setlength{\parindent}{1cm}#1 \Qed}

% Things Lie
\newcommand{\kb}{\mathfrak b}
\newcommand{\kg}{\mathfrak g}
\newcommand{\kh}{\mathfrak h}
\newcommand{\kn}{\mathfrak n}
\newcommand{\ku}{\mathfrak u}
\newcommand{\kz}{\mathfrak z}
\DeclareMathOperator{\Ext}{Ext} % Ext functor
\DeclareMathOperator{\Tor}{Tor} % Tor functor
\newcommand{\gl}{\opname{\mathfrak{gl}}} % frak gl group
\renewcommand{\sl}{\opname{\mathfrak{sl}}} % frak sl group chktex 6

% More script letters etc.
\newcommand{\SA}{\mathcal A}
\newcommand{\SB}{\mathcal B}
\newcommand{\SC}{\mathcal C}
\newcommand{\SF}{\mathcal F}
\newcommand{\SG}{\mathcal G}
\newcommand{\SH}{\mathcal H}
\newcommand{\OO}{\mathcal O}

\newcommand{\SCA}{\mathscr A}
\newcommand{\SCB}{\mathscr B}
\newcommand{\SCC}{\mathscr C}
\newcommand{\SCD}{\mathscr D}
\newcommand{\SCE}{\mathscr E}
\newcommand{\SCF}{\mathscr F}
\newcommand{\SCG}{\mathscr G}
\newcommand{\SCH}{\mathscr H}

% Mathfrak primes
\newcommand{\km}{\mathfrak m}
\newcommand{\kp}{\mathfrak p}
\newcommand{\kq}{\mathfrak q}

% number sets
\newcommand{\RR}[1][]{\ensuremath{\ifstrempty{#1}{\mathbb{R}}{\mathbb{R}^{#1}}}}
\newcommand{\NN}[1][]{\ensuremath{\ifstrempty{#1}{\mathbb{N}}{\mathbb{N}^{#1}}}}
\newcommand{\ZZ}[1][]{\ensuremath{\ifstrempty{#1}{\mathbb{Z}}{\mathbb{Z}^{#1}}}}
\newcommand{\QQ}[1][]{\ensuremath{\ifstrempty{#1}{\mathbb{Q}}{\mathbb{Q}^{#1}}}}
\newcommand{\CC}[1][]{\ensuremath{\ifstrempty{#1}{\mathbb{C}}{\mathbb{C}^{#1}}}}
\newcommand{\PP}[1][]{\ensuremath{\ifstrempty{#1}{\mathbb{P}}{\mathbb{P}^{#1}}}}
\newcommand{\HH}[1][]{\ensuremath{\ifstrempty{#1}{\mathbb{H}}{\mathbb{H}^{#1}}}}
\newcommand{\FF}[1][]{\ensuremath{\ifstrempty{#1}{\mathbb{F}}{\mathbb{F}^{#1}}}}
% expected value
\newcommand{\EE}{\ensuremath{\mathbb{E}}}
\newcommand{\charin}{\text{ char }}
\DeclareMathOperator{\sign}{sign}
\DeclareMathOperator{\Aut}{Aut}
\DeclareMathOperator{\Inn}{Inn}
\DeclareMathOperator{\Syl}{Syl}
\DeclareMathOperator{\Gal}{Gal}
\DeclareMathOperator{\GL}{GL} % General linear group
\DeclareMathOperator{\SL}{SL} % Special linear group

%---------------------------------------
% BlackBoard Math Fonts :-
%---------------------------------------

%Captital Letters
\newcommand{\bbA}{\mathbb{A}}	\newcommand{\bbB}{\mathbb{B}}
\newcommand{\bbC}{\mathbb{C}}	\newcommand{\bbD}{\mathbb{D}}
\newcommand{\bbE}{\mathbb{E}}	\newcommand{\bbF}{\mathbb{F}}
\newcommand{\bbG}{\mathbb{G}}	\newcommand{\bbH}{\mathbb{H}}
\newcommand{\bbI}{\mathbb{I}}	\newcommand{\bbJ}{\mathbb{J}}
\newcommand{\bbK}{\mathbb{K}}	\newcommand{\bbL}{\mathbb{L}}
\newcommand{\bbM}{\mathbb{M}}	\newcommand{\bbN}{\mathbb{N}}
\newcommand{\bbO}{\mathbb{O}}	\newcommand{\bbP}{\mathbb{P}}
\newcommand{\bbQ}{\mathbb{Q}}	\newcommand{\bbR}{\mathbb{R}}
\newcommand{\bbS}{\mathbb{S}}	\newcommand{\bbT}{\mathbb{T}}
\newcommand{\bbU}{\mathbb{U}}	\newcommand{\bbV}{\mathbb{V}}
\newcommand{\bbW}{\mathbb{W}}	\newcommand{\bbX}{\mathbb{X}}
\newcommand{\bbY}{\mathbb{Y}}	\newcommand{\bbZ}{\mathbb{Z}}

%---------------------------------------
% MathCal Fonts :-
%---------------------------------------

%Captital Letters
\newcommand{\mcA}{\mathcal{A}}	\newcommand{\mcB}{\mathcal{B}}
\newcommand{\mcC}{\mathcal{C}}	\newcommand{\mcD}{\mathcal{D}}
\newcommand{\mcE}{\mathcal{E}}	\newcommand{\mcF}{\mathcal{F}}
\newcommand{\mcG}{\mathcal{G}}	\newcommand{\mcH}{\mathcal{H}}
\newcommand{\mcI}{\mathcal{I}}	\newcommand{\mcJ}{\mathcal{J}}
\newcommand{\mcK}{\mathcal{K}}	\newcommand{\mcL}{\mathcal{L}}
\newcommand{\mcM}{\mathcal{M}}	\newcommand{\mcN}{\mathcal{N}}
\newcommand{\mcO}{\mathcal{O}}	\newcommand{\mcP}{\mathcal{P}}
\newcommand{\mcQ}{\mathcal{Q}}	\newcommand{\mcR}{\mathcal{R}}
\newcommand{\mcS}{\mathcal{S}}	\newcommand{\mcT}{\mathcal{T}}
\newcommand{\mcU}{\mathcal{U}}	\newcommand{\mcV}{\mathcal{V}}
\newcommand{\mcW}{\mathcal{W}}	\newcommand{\mcX}{\mathcal{X}}
\newcommand{\mcY}{\mathcal{Y}}	\newcommand{\mcZ}{\mathcal{Z}}


%---------------------------------------
% Bold Math Fonts :-
%---------------------------------------

%Captital Letters
\newcommand{\bmA}{\boldsymbol{A}}	\newcommand{\bmB}{\boldsymbol{B}}
\newcommand{\bmC}{\boldsymbol{C}}	\newcommand{\bmD}{\boldsymbol{D}}
\newcommand{\bmE}{\boldsymbol{E}}	\newcommand{\bmF}{\boldsymbol{F}}
\newcommand{\bmG}{\boldsymbol{G}}	\newcommand{\bmH}{\boldsymbol{H}}
\newcommand{\bmI}{\boldsymbol{I}}	\newcommand{\bmJ}{\boldsymbol{J}}
\newcommand{\bmK}{\boldsymbol{K}}	\newcommand{\bmL}{\boldsymbol{L}}
\newcommand{\bmM}{\boldsymbol{M}}	\newcommand{\bmN}{\boldsymbol{N}}
\newcommand{\bmO}{\boldsymbol{O}}	\newcommand{\bmP}{\boldsymbol{P}}
\newcommand{\bmQ}{\boldsymbol{Q}}	\newcommand{\bmR}{\boldsymbol{R}}
\newcommand{\bmS}{\boldsymbol{S}}	\newcommand{\bmT}{\boldsymbol{T}}
\newcommand{\bmU}{\boldsymbol{U}}	\newcommand{\bmV}{\boldsymbol{V}}
\newcommand{\bmW}{\boldsymbol{W}}	\newcommand{\bmX}{\boldsymbol{X}}
\newcommand{\bmY}{\boldsymbol{Y}}	\newcommand{\bmZ}{\boldsymbol{Z}}
%Small Letters
\newcommand{\bma}{\boldsymbol{a}}	\newcommand{\bmb}{\boldsymbol{b}}
\newcommand{\bmc}{\boldsymbol{c}}	\newcommand{\bmd}{\boldsymbol{d}}
\newcommand{\bme}{\boldsymbol{e}}	\newcommand{\bmf}{\boldsymbol{f}}
\newcommand{\bmg}{\boldsymbol{g}}	\newcommand{\bmh}{\boldsymbol{h}}
\newcommand{\bmi}{\boldsymbol{i}}	\newcommand{\bmj}{\boldsymbol{j}}
\newcommand{\bmk}{\boldsymbol{k}}	\newcommand{\bml}{\boldsymbol{l}}
\newcommand{\bmm}{\boldsymbol{m}}	\newcommand{\bmn}{\boldsymbol{n}}
\newcommand{\bmo}{\boldsymbol{o}}	\newcommand{\bmp}{\boldsymbol{p}}
\newcommand{\bmq}{\boldsymbol{q}}	\newcommand{\bmr}{\boldsymbol{r}}
\newcommand{\bms}{\boldsymbol{s}}	\newcommand{\bmt}{\boldsymbol{t}}
\newcommand{\bmu}{\boldsymbol{u}}	\newcommand{\bmv}{\boldsymbol{v}}
\newcommand{\bmw}{\boldsymbol{w}}	\newcommand{\bmx}{\boldsymbol{x}}
\newcommand{\bmy}{\boldsymbol{y}}	\newcommand{\bmz}{\boldsymbol{z}}

%---------------------------------------
% Scr Math Fonts :-
%---------------------------------------

\newcommand{\sA}{{\mathscr{A}}}   \newcommand{\sB}{{\mathscr{B}}}
\newcommand{\sC}{{\mathscr{C}}}   \newcommand{\sD}{{\mathscr{D}}}
\newcommand{\sE}{{\mathscr{E}}}   \newcommand{\sF}{{\mathscr{F}}}
\newcommand{\sG}{{\mathscr{G}}}   \newcommand{\sH}{{\mathscr{H}}}
\newcommand{\sI}{{\mathscr{I}}}   \newcommand{\sJ}{{\mathscr{J}}}
\newcommand{\sK}{{\mathscr{K}}}   \newcommand{\sL}{{\mathscr{L}}}
\newcommand{\sM}{{\mathscr{M}}}   \newcommand{\sN}{{\mathscr{N}}}
\newcommand{\sO}{{\mathscr{O}}}   \newcommand{\sP}{{\mathscr{P}}}
\newcommand{\sQ}{{\mathscr{Q}}}   \newcommand{\sR}{{\mathscr{R}}}
\newcommand{\sS}{{\mathscr{S}}}   \newcommand{\sT}{{\mathscr{T}}}
\newcommand{\sU}{{\mathscr{U}}}   \newcommand{\sV}{{\mathscr{V}}}
\newcommand{\sW}{{\mathscr{W}}}   \newcommand{\sX}{{\mathscr{X}}}
\newcommand{\sY}{{\mathscr{Y}}}   \newcommand{\sZ}{{\mathscr{Z}}}


%---------------------------------------
% Math Fraktur Font
%---------------------------------------

%Captital Letters
\newcommand{\mfA}{\mathfrak{A}}	\newcommand{\mfB}{\mathfrak{B}}
\newcommand{\mfC}{\mathfrak{C}}	\newcommand{\mfD}{\mathfrak{D}}
\newcommand{\mfE}{\mathfrak{E}}	\newcommand{\mfF}{\mathfrak{F}}
\newcommand{\mfG}{\mathfrak{G}}	\newcommand{\mfH}{\mathfrak{H}}
\newcommand{\mfI}{\mathfrak{I}}	\newcommand{\mfJ}{\mathfrak{J}}
\newcommand{\mfK}{\mathfrak{K}}	\newcommand{\mfL}{\mathfrak{L}}
\newcommand{\mfM}{\mathfrak{M}}	\newcommand{\mfN}{\mathfrak{N}}
\newcommand{\mfO}{\mathfrak{O}}	\newcommand{\mfP}{\mathfrak{P}}
\newcommand{\mfQ}{\mathfrak{Q}}	\newcommand{\mfR}{\mathfrak{R}}
\newcommand{\mfS}{\mathfrak{S}}	\newcommand{\mfT}{\mathfrak{T}}
\newcommand{\mfU}{\mathfrak{U}}	\newcommand{\mfV}{\mathfrak{V}}
\newcommand{\mfW}{\mathfrak{W}}	\newcommand{\mfX}{\mathfrak{X}}
\newcommand{\mfY}{\mathfrak{Y}}	\newcommand{\mfZ}{\mathfrak{Z}}
%Small Letters
\newcommand{\mfa}{\mathfrak{a}}	\newcommand{\mfb}{\mathfrak{b}}
\newcommand{\mfc}{\mathfrak{c}}	\newcommand{\mfd}{\mathfrak{d}}
\newcommand{\mfe}{\mathfrak{e}}	\newcommand{\mff}{\mathfrak{f}}
\newcommand{\mfg}{\mathfrak{g}}	\newcommand{\mfh}{\mathfrak{h}}
\newcommand{\mfi}{\mathfrak{i}}	\newcommand{\mfj}{\mathfrak{j}}
\newcommand{\mfk}{\mathfrak{k}}	\newcommand{\mfl}{\mathfrak{l}}
\newcommand{\mfm}{\mathfrak{m}}	\newcommand{\mfn}{\mathfrak{n}}
\newcommand{\mfo}{\mathfrak{o}}	\newcommand{\mfp}{\mathfrak{p}}
\newcommand{\mfq}{\mathfrak{q}}	\newcommand{\mfr}{\mathfrak{r}}
\newcommand{\mfs}{\mathfrak{s}}	\newcommand{\mft}{\mathfrak{t}}
\newcommand{\mfu}{\mathfrak{u}}	\newcommand{\mfv}{\mathfrak{v}}
\newcommand{\mfw}{\mathfrak{w}}	\newcommand{\mfx}{\mathfrak{x}}
\newcommand{\mfy}{\mathfrak{y}}	\newcommand{\mfz}{\mathfrak{z}}


\title{\Huge{Math 2700.009}\\Problem Set 08}
\author{\huge{Ezekiel Berumen}}
\date{19 March 2024}

\begin{document}

\maketitle
\newpage

\qs{}{ Consider the following bases for $\mathbb{R}^4$
$$
\mathfrak{B}=\left\langle\begin{bmatrix}
1 \\
1 \\
1 \\
1
\end{bmatrix},
\begin{bmatrix}
1 \\
1 \\
1 \\
0
\end{bmatrix},
\begin{bmatrix}
1 \\
1 \\
0 \\
0
\end{bmatrix},
\begin{bmatrix}
1 \\
0 \\
0 \\
0
\end{bmatrix}\right\rangle \quad \text { and } \quad \mathfrak{D}=\left\langle
\begin{bmatrix}
2 \\
1 \\
0 \\
0
\end{bmatrix},
\begin{bmatrix}
3 \\
2 \\
0 \\
0
\end{bmatrix},
\begin{bmatrix}
0 \\
0 \\
3 \\
2
\end{bmatrix},
\begin{bmatrix}
0 \\
0 \\
4 \\
3
\end{bmatrix}\right\rangle
$$
and let $\mathfrak{E}=\left\langle\vec{e}_1, \vec{e}_2, \vec{e}_3, \vec{e}_4\right\rangle$ be the standard basis for $\mathbb{R}^4$. Compute each of the following: \\
(a) $P_{\mathfrak{B} \rightarrow \mathfrak{E}}$ \\
(b) $P_{\mathfrak{D} \rightarrow \mathfrak{E}}$ \\
(c) $P_{\mathfrak{E} \rightarrow \mathfrak{B}}$ \\
(d) $P_{\mathfrak{E} \rightarrow \mathfrak{D}}$ \\
(e) $P_{\mathfrak{B} \rightarrow \mathfrak{D}}$ \\
(f) $P_{\mathfrak{D} \rightarrow \mathfrak{B}}$ }
\begin{note}
\begin{itemize}
\item If
$\left\{\begin{array}{l}
\mathfrak{E} = \langle \vec{e_1},\ldots,\vec{e_n} \rangle \\
\mathfrak{B} = \langle \vec{b_1},\ldots,\vec{b_n} \rangle
\end{array}
\right.$, then $P_{\mathfrak{B} \rightarrow \mathfrak{E}}[\vec{x}]_{\mathfrak{B}} = [\vec{x}]_{\mathfrak{E}} = \vec{x}$ \\
\item
$P_{\mathfrak{B} \rightarrow \mathfrak{E}} = 
\left[
  \begin{array}{cccc}
    \vrule & \vrule & & \vrule\\
    \vec{b_1} & \vec{b_2} & \ldots & \vec{b_n} \\
    \vrule & \vrule & & \vrule 
  \end{array}
\right]$ Where $\mathfrak{E}$ is the standard basis.
\item $(P_{\mathfrak{B} \rightarrow \mathfrak{E}})^{-1} = P_{\mathfrak{E}\rightarrow\mathfrak{B}}$
\item If $\mathfrak{B}_1,\mathfrak{B}_2,$ and $\mathfrak{B}_3$ are bases, then $P_{\mathfrak{B}_2\rightarrow\mathfrak{B}_3}P_{\mathfrak{B}_1\rightarrow\mathfrak{B}_2}[\vec{x}]_{\mathfrak{B}_1}=[\vec{x}]_{\mathfrak{B}_3}$
\end{itemize}
\end{note}
\sol \\
(a) $P_{\mathfrak{B} \rightarrow \mathfrak{E}} =
\begin{bmatrix}
1 & 1 & 1 & 1 \\
1 & 1 &1 & 0 \\
1 & 1 & 0 & 0 \\
1 & 0 & 0 & 0
\end{bmatrix} $ \\
\noindent
(b) $P_{\mathfrak{D} \rightarrow \mathfrak{E}} =
\begin{bmatrix}
2 & 3 & 0 & 0 \\
1 & 2 & 0 & 0 \\
0 & 0 & 3 & 4 \\
0 & 0 & 2 & 3
\end{bmatrix} $ \\
\noindent
(c) $P_{\mathfrak{E} \rightarrow \mathfrak{B}} =
\begin{bmatrix}
1 & 1 & 1 & 1 \\
1 & 1 &1 & 0 \\
1 & 1 & 0 & 0 \\
1 & 0 & 0 & 0
\end{bmatrix} ^{-1}$
\begin{align*}
	&\left[\begin{array}{cccc|cccc}
		1 & 1 & 1 & 1 & 1 & 0 & 0 & 0 \\
		1 & 1 & 1 & 0 & 0 & 1 & 0 & 0 \\
		1 & 1 & 0 & 0 & 0 & 0 & 1 & 0 \\
		1 & 0 & 0 & 0 & 0 & 0 & 0 & 1
	\end{array}\right]
	&& \begin{aligned} & R_1 \leftrightarrow R_4 \\ & R_2 \leftrightarrow R_3 \end{aligned}
	&\left[\begin{array}{cccc|cccc}
		1 & 0 & 0 & 0 & 0 & 0 & 0 & 1 \\
		1 & 1 & 0 & 0 & 0 & 0 & 1 & 0 \\
		1 & 1 & 1 & 0 & 0 & 1 & 0 & 0 \\
		1 & 1 & 1 & 1 & 1 & 0 & 0 & 0
	\end{array}\right]
	&& R_4 - R_3 \rightarrow R_4 \\
	&\left[\begin{array}{cccc|cccc}
		1 & 0 & 0 & 0 & 0 & 0 & 0 & 1 \\
		1 & 1 & 0 & 0 & 0 & 0 & 1 & 0 \\
		1 & 1 & 1 & 0 & 0 & 1 & 0 & 0 \\
		0 & 0 & 0 & 1 & 1 & -1 & 0 & 0
	\end{array}\right]
	&& R_3 - R_2 \rightarrow R_3
	&\left[\begin{array}{cccc|cccc}
		1 & 0 & 0 & 0 & 0 & 0 & 0 & 1 \\
		1 & 1 & 0 & 0 & 0 & 0 & 1 & 0 \\
		0 & 0 & 1 & 0 & 0 & 1 & -1 & 0 \\
		0 & 0 & 0 & 1 & 1 & -1 & 0 & 0
	\end{array}\right]
	&& R_2 - R_1 \rightarrow R_2 \\
\end{align*}
\[
	\left[\begin{array}{cccc|cccc}
		1 & 0 & 0 & 0 & 0 & 0 & 0 & 1 \\
		0 & 1 & 0 & 0 & 0 & 0 & 1 & -1 \\
		0 & 0 & 1 & 0 & 0 & 1 & -1 & 0 \\
		0 & 0 & 0 & 1 & 1 & -1 & 0 & 0
	\end{array}\right]
	\implies P_{\mathfrak{E}\rightarrow\mathfrak{B}} = \begin{bmatrix} 0 & 0 & 0 & 1 \\ 0 & 0 & 1 & -1 \\ 0 & 1 & -1 & 0 \\ 1 & -1 & 0 & 0\end{bmatrix}
\]\\
\noindent
(d) $P_{\mathfrak{E}\rightarrow\mathfrak{D}} =
\begin{bmatrix}
2 & 3 & 0 & 0 \\
1 & 2 & 0 & 0 \\
0 & 0 & 3 & 4 \\
0 & 0 & 2 & 3
\end{bmatrix} ^{-1}$
\begin{align*}
	&\left[\begin{array}{cccc|cccc}
		2 & 3 & 0 & 0 & 1 & 0 & 0 & 0 \\
		1 & 2 & 0 & 0 & 0 & 1 & 0 & 0 \\
		0 & 0 & 3 & 4 & 0 & 0 & 1 & 0 \\
		0 & 0 & 2 & 3 & 0 & 0 & 0 & 1
	\end{array}\right]
	&& R_1 \leftrightarrow R_2
	&\left[\begin{array}{cccc|cccc}
		1 & 2 & 0 & 0 & 0 & 1 & 0 & 0 \\
		2 & 3 & 0 & 0 & 1 & 0 & 0 & 0 \\
		0 & 0 & 3 & 4 & 0 & 0 & 1 & 0 \\
		0 & 0 & 2 & 3 & 0 & 0 & 0 & 1
	\end{array}\right]
	&& R_2 - 2R_1 \rightarrow R_2 \\
	&\left[\begin{array}{cccc|cccc}
		1 & 2 & 0 & 0 & 0 & 1 & 0 & 0 \\
		0 & -1 & 0 & 0 & 1 & -2 & 0 & 0 \\
		0 & 0 & 3 & 4 & 0 & 0 & 1 & 0 \\
		0 & 0 & 2 & 3 & 0 & 0 & 0 & 1
	\end{array}\right]
	&& R_1 + 2R_2 \rightarrow R_1
	&\left[\begin{array}{cccc|cccc}
		1 & 0 & 0 & 0 & 2 & -3 & 0 & 0 \\
		0 & -1 & 0 & 0 & 1 & -2 & 0 & 0 \\
		0 & 0 & 3 & 4 & 0 & 0 & 1 & 0 \\
		0 & 0 & 2 & 3 & 0 & 0 & 0 & 1
	\end{array}\right]
	&& -R_2 \rightarrow R_2 \\
	&\left[\begin{array}{cccc|cccc}
		1 & 0 & 0 & 0 & 2 & -3 & 0 & 0 \\
		0 & 1 & 0 & 0 & -1 & 2 & 0 & 0 \\
		0 & 0 & 3 & 4 & 0 & 0 & 1 & 0 \\
		0 & 0 & 2 & 3 & 0 & 0 & 0 & 1
	\end{array}\right]
	&& \frac{1}{3}R_3\rightarrow R_3
	&\left[\begin{array}{cccc|cccc}
		1 & 0 & 0 & 0 & 2 & -3 & 0 & 0 \\
		0 & 1 & 0 & 0 & -1 & 2 & 0 & 0 \\
		0 & 0 & 1 & \frac{4}{3} & 0 & 0 & \frac{1}{3} & 0 \\
		0 & 0 & 2 & 3 & 0 & 0 & 0 & 1
	\end{array}\right]
	&& R_4 - 2R_3 \rightarrow R_4 \\
	&\left[\begin{array}{cccc|cccc}
		1 & 0 & 0 & 0 & 2 & -3 & 0 & 0 \\
		0 & 1 & 0 & 0 & -1 & 2 & 0 & 0 \\
		0 & 0 & 1 & \frac{4}{3} & 0 & 0 & \frac{1}{3} & 0 \\
		0 & 0 & 0 & \frac{1}{3} & 0 & 0 & -\frac{2}{3} & 1
	\end{array}\right]
	&& R_3 - 4R_4 \rightarrow R_3
	&\left[\begin{array}{cccc|cccc}
		1 & 0 & 0 & 0 & 2 & -3 & 0 & 0 \\
		0 & 1 & 0 & 0 & -1 & 2 & 0 & 0 \\
		0 & 0 & 1 & 0 & 0 & 0 & 3 & -4 \\
		0 & 0 & 0 & \frac{1}{3} & 0 & 0 & -\frac{2}{3} & 1
	\end{array}\right]
	&& 3R_4 \rightarrow R_4
\end{align*}
\[
	\left[\begin{array}{cccc|cccc}
		1 & 0 & 0 & 0 & 2 & -3 & 0 & 0 \\
		0 & 1 & 0 & 0 & -1 & 2 & 0 & 0 \\
		0 & 0 & 1 & 0 & 0 & 0 & 3 & -4 \\
		0 & 0 & 0 & 1 & 0 & 0 & -2 & 3
	\end{array}\right]
	\implies P_{\mathfrak{E}\rightarrow\mathfrak{D}} = \begin{bmatrix}  2 & -3 & 0 & 0 \\ -1 & 2 & 0 & 0 \\ 0 & 0 & 3 & -4 \\ 0 & 0 & -2 & 3 \end{bmatrix}
\]
(e) $P_{\mathfrak{B}\rightarrow\mathfrak{D}} = P_{\mathfrak{E}\rightarrow\mathfrak{D}}P_{\mathfrak{B}\rightarrow\mathfrak{E}} =
\begin{bmatrix}  
2 & -3 & 0 & 0 \\
-1 & 2 & 0 & 0 \\
0 & 0 & 3 & -4 \\
0 & 0 & -2 & 3
\end{bmatrix}
\begin{bmatrix}
1 & 1 & 1 & 1 \\
1 & 1 &1 & 0 \\
1 & 1 & 0 & 0 \\
1 & 0 & 0 & 0
\end{bmatrix} =
\begin{bmatrix}
-1 & -1 & -1 & 2 \\
1 & 1 & 1 & -1 \\
-1 & 3 & 0 & 0 \\
1 & -2 & 0 & 0
\end{bmatrix}
$ \\
(f) $P_{\mathfrak{D}\rightarrow\mathfrak{B}} = P_{\mathfrak{E}\rightarrow\mathfrak{B}}P_{\mathfrak{D}\rightarrow\mathfrak{E}} =
 \begin{bmatrix} 
0 & 0 & 0 & 1 \\
0 & 0 & 1 & -1 \\
0 & 1 & -1 & 0 \\
1 & -1 & 0 & 0\end{bmatrix}
\begin{bmatrix}
2 & 3 & 0 & 0 \\
1 & 2 & 0 & 0 \\
0 & 0 & 3 & 4 \\
0 & 0 & 2 & 3
\end{bmatrix} =
\begin{bmatrix}
0 & 0 & 2 & 3 \\
0 & 0 & 1 & 1 \\
1 & 2 & -3 & -4 \\
1 & 1 & 0 & 0
\end{bmatrix}
$


\qs{}{The point of this problem is to see the effect of changing the order of the basis and the effect on the coordinate vectors. Let $\mathfrak{D}$ and $\mathfrak{E}$ be as in problem 1 and let
$$
\mathfrak{F}=\left\langle\left[\begin{array}{l}
0 \\
0 \\
3 \\
2
\end{array}\right],\left[\begin{array}{l}
3 \\
2 \\
0 \\
0
\end{array}\right],\left[\begin{array}{l}
0 \\
0 \\
4 \\
3
\end{array}\right],\left[\begin{array}{l}
2 \\
1 \\
0 \\
0
\end{array}\right]\right\rangle .
$$
(a) What are $P_{\mathfrak{E} \rightarrow \mathfrak{F}}$ and $P_{\mathfrak{D} \rightarrow \mathfrak{F}}$ ? (note you have $P_{\mathfrak{D} \rightarrow \mathfrak{E}}$ from problem 1) \\
(b) Suppose $\vec{x} \in \mathbb{R}^4$ had
$$
[\vec{x}]_{\mathfrak{D}}=\left[\begin{array}{l}
a \\
b \\
c \\
d
\end{array}\right]
$$

Use $P_{\mathfrak{D} \rightarrow \mathfrak{F}}$ to find $[\vec{x}]_{\mathfrak{F}}$.}
\sol \\
(a) $P_{\mathfrak{E}\rightarrow\mathfrak{F}} = 
\begin{bmatrix} 
0 & 3 & 0 & 2 \\ 
0 & 2 & 0 & 1 \\
3 & 0 & 4 & 0 \\ 
2 & 0 & 3 & 0 \end{bmatrix}^{-1}$
\begin{align*}
	&\left[\begin{array}{cccc|cccc}
		0 & 3 & 0 & 2 & 1 & 0 & 0 & 0 \\
		0 & 2 & 0 & 1 & 0 & 1 & 0 & 0 \\
		3 & 0 & 4 & 0 & 0 & 0 & 1 & 0 \\
		2 & 0 & 3 & 0 & 0 & 0 & 0 & 1
	\end{array}\right]
	&& R_1 \leftrightarrow R_4
	&\left[\begin{array}{cccc|cccc}
		2 & 0 & 3 & 0 & 0 & 0 & 0 & 1 \\
		0 & 2 & 0 & 1 & 0 & 1 & 0 & 0 \\
		3 & 0 & 4 & 0 & 0 & 0 & 1 & 0 \\
		0 & 3 & 0 & 2 & 1 & 0 & 0 & 0
	\end{array}\right]
	&& \begin{aligned} & R_3 - R_1 \rightarrow R_3 \\ & R_4 - R_2 \rightarrow R_4 \end{aligned} \\
	&\left[\begin{array}{cccc|cccc}
		2 & 0 & 3 & 0 & 0 & 0 & 0 & 1 \\
		0 & 2 & 0 & 1 & 0 & 1 & 0 & 0 \\
		1 & 0 & 1 & 0 & 0 & 0 & 1 & -1 \\
		0 & 1 & 0 & 1 & 1 & -1 & 0 & 0
	\end{array}\right]
	&& \begin{aligned} & R_1 - 2R_3 \rightarrow R_1 \\ & R_2 - R_4 \rightarrow R_2 \end{aligned}
	&\left[\begin{array}{cccc|cccc}
		0 & 0 & 1 & 0 & 0 & 0 & -2 & 3 \\
		0 & 1 & 0 & 0 & -1 & 2 & 0 & 0 \\
		1 & 0 & 1 & 0 & 0 & 0 & 1 & -1 \\
		0 & 1 & 0 & 1 & 1 & -1 & 0 & 0
	\end{array}\right]
	&& \begin{aligned} & R_4 - R_2 \rightarrow R_4 \\ & R_3 - R_1 \rightarrow R_3 \end{aligned} \\
	&\left[\begin{array}{cccc|cccc}
		0 & 0 & 1 & 0 & 0 & 0 & -2 & 3 \\
		0 & 1 & 0 & 0 & -1 & 2 & 0 & 0 \\
		1 & 0 & 0 & 0 & 0 & 0 & 3 & -4 \\
		0 & 0 & 0 & 1 & 2 & -3 & 0 & 0
	\end{array}\right]
	&& R_1 \leftrightarrow R_3
	&\left[\begin{array}{cccc|cccc}
		1 & 0 & 0 & 0 & 0 & 0 & 3 & -4 \\
		0 & 1 & 0 & 0 & -1 & 2 & 0 & 0 \\
		0 & 0 & 1 & 0 & 0 & 0 & -2 & 3 \\
		0 & 0 & 0 & 1 & 2 & -3 & 0 & 0
	\end{array}\right]
\end{align*}
\[
P_{\mathfrak{E}\rightarrow\mathfrak{F}} = 
\begin{bmatrix} 
0 & 0 & 3 & -4 \\
-1 & 2 & 0 & 0 \\
0 & 0 & -2 & 3 \\
2 & -3 & 0 & 0
\end{bmatrix} \text{ and } P_{\mathfrak{D}\rightarrow\mathfrak{F}} = P_{\mathfrak{E}\rightarrow\mathfrak{F}}P_{\mathfrak{D}\rightarrow\mathfrak{E}}
\]
So then, $P_{\mathfrak{D}\rightarrow\mathfrak{F}} =
\begin{bmatrix} 
0 & 0 & 3 & -4 \\
-1 & 2 & 0 & 0 \\
0 & 0 & -2 & 3 \\
2 & -3 & 0 & 0
\end{bmatrix} \begin{bmatrix}
2 & 3 & 0 & 0 \\
1 & 2 & 0 & 0 \\
0 & 0 & 3 & 4 \\
0 & 0 & 2 & 3
\end{bmatrix} = \begin{bmatrix}
0 & 0 & 1 & 0 \\
0 & 1 & 0 & 0 \\
0 & 0 & 0 & 1 \\
1 & 0 & 0 & 0
\end{bmatrix}$ \\
\\
\noindent
(b) To find $[\vec{x}]_{\mathfrak{F}}$, then we we must find $P_{\mathfrak{E}\rightarrow\mathfrak{F}}P_{\mathfrak{D}\rightarrow\mathfrak{E}}[\vec{x}]_{\mathfrak{D}}$
\[
\begin{aligned}
[\vec{x}]_{\mathfrak{F}} & = P_{\mathfrak{E}\rightarrow\mathfrak{F}}P_{\mathfrak{D}\rightarrow\mathfrak{E}}[\vec{x}]_{\mathfrak{D}} \\
& = \begin{bmatrix} 
0 & 0 & 3 & -4 \\
-1 & 2 & 0 & 0 \\
0 & 0 & -2 & 3 \\
2 & -3 & 0 & 0
\end{bmatrix} \begin{bmatrix}
2 & 3 & 0 & 0 \\
1 & 2 & 0 & 0 \\
0 & 0 & 3 & 4 \\
0 & 0 & 2 & 3
\end{bmatrix} \begin{bmatrix}
a \\ b \\ c \\ d
\end{bmatrix} \\
& = \begin{bmatrix}
0 & 0 & 1 & 0 \\
0 & 1 & 0 & 0 \\
0 & 0 & 0 & 1 \\
1 & 0 & 0 & 0
\end{bmatrix} \begin{bmatrix}
a \\ b \\ c \\ d
\end{bmatrix} \\
[\vec{x}]_{\mathfrak{F}} & = \begin{bmatrix}
c \\ b \\ d \\ a
\end{bmatrix}
\end{aligned}
\]
\qs{}{
In $\mathbb{R}^3$, let
$$
\vec{e}_1=\left[\begin{array}{l}
1 \\
0 \\
0
\end{array}\right] \quad \text { and } \quad \vec{e}_2=\left[\begin{array}{l}
0 \\
1 \\
0
\end{array}\right]
$$
and let $W=\operatorname{span}\left(\vec{e}_1, \vec{e}_2\right)$. Let $T: \mathbb{R}^2 \rightarrow W$ be defined by
$$
T\left(\left[\begin{array}{l}
x \\
y
\end{array}\right]\right)=\left[\begin{array}{l}
x \\
y \\
0
\end{array}\right] .
$$
(a) What is $\operatorname{dim}(W)$ ? \\
(b) What is $\operatorname{ker}(T)$ ? and what is $\operatorname{ran}(T)$ ? \\
(c) Is $T$ an isomorphism? (Recall that $T$ is an isomorphism if and only if $\operatorname{ker}(T)=\{\overrightarrow{0}\}$ and $\operatorname{ran}(T)=W$. \\
(d) Find a matrix $A \in \mathbb{R}^{2 \times 2}$ so that
$$
[T(\vec{x})]_{\mathfrak{B}}=A \vec{x}
$$
where $\mathfrak{B}=\left\langle\vec{e}_1, \vec{e}_2\right\rangle$ is the natural basis for $W$}
\sol \\
(a) The $\operatorname{dim}(W)$ is the number of linearly independent vectors that also span $W$. That is to say the dimension of $W$ is the number of vectors in a basis for $W$. Since $\vec{e_1}$ and $\vec{e_2}$ are linearly independent, and they span $W$, then they also form a basis for $W$, so by definition of dimension, it must be 2. \\
\\
\noindent (b) The $\operatorname{ker}(T)$ consists of the vectors from $\mathbb{R}^2$ that map to the zero vector of $W$ which is the same as $\mathbb{R}^3$. If we set $$
T(\begin{bmatrix} x \\ y \end{bmatrix}) = \begin{bmatrix} 0 \\ 0 \\ 0 \end{bmatrix}
$$ we can see that the only solution which results in the zero vector is when $x = y = 0$. This means that $$\operatorname{ker}(T) = \left\{\begin{bmatrix} 0 \\ 0 \end{bmatrix}\right\} = \left\{\vec{0}\right\}$$
The $\operatorname{ran}(T)$ consists of all possible outputs of the linear transformation. The linear transformation $T$ serves to map vectors from $\mathbb{R}^2$ to $W$. Since $W$ is a subspace generated by taking the span of $\vec{e_1}$ and $\vec{e_2}$, then it is sufficient to say that the range of $T$ covers all vectors from $W$. Thus we can say that $\operatorname{ran}(T) = W$. \\
\\
\noindent (c) As discovered in (b), We can say that T is an isomorphism, since it meets the biconditional statement. \\
\\
\noindent (d) To find the matrix $A\in\mathbb{R}^{2\times2}$, we can use some principles. Since the ordered basis we will be using in this example is standardized, then a useful property can be used, that is that $[\vec{x}]_{\mathfrak{E}} = \vec{x}$. Thus we can express our equation as $[T(\vec{x})]_{\mathfrak{B}} = A[\vec{x}]_{\mathfrak{B}}$ since $\mathfrak{B}$ is effectively the standard basis. We can find the matrix by taking the columns as the $\mathfrak{B}$-coordinate vectors of the calculated values from T so that we have
$$
A = \left[
  \begin{array}{cc}
    \vrule & \vrule \\
    \left[T(e_{1})\right]_{\mathfrak{B}} & \left[T(e_{2})\right]_{\mathfrak{B}}\\
    \vrule & \vrule 
  \end{array}
\right] = \begin{bmatrix} 1 & 0 \\ 0 & 1 \end{bmatrix}
$$

\qs{}{ Let
$$
\mathfrak{B}=\left\langle\left[\begin{array}{l}
2 \\
2 \\
0 \\
0
\end{array}\right],\left[\begin{array}{l}
3 \\
1 \\
0 \\
0
\end{array}\right],\left[\begin{array}{l}
0 \\
0 \\
1 \\
2
\end{array}\right],\left[\begin{array}{l}
0 \\
0 \\
1 \\
1
\end{array}\right]\right\rangle
$$
and $T: \mathbb{R}^4 \rightarrow \mathbb{R}^4$ be given by
$$
T(\vec{x})=\left[\begin{array}{cccc}
5 / 2 & 3 / 2 & 0 & 0 \\
-1 & 1 / 2 & 0 & 0 \\
0 & 0 & 5 / 4 & 1 / 4 \\
0 & 0 & -1 / 2 & 1 / 2
\end{array}\right] \vec{x} .
$$

Find the $\mathfrak{B}$-matrix of $T$, that is, find a matrix $A \in \mathbb{R}^{4 \times 4}$ so that
$$
[T(\vec{x})]_{\mathfrak{B}}=A[\vec{x}]_{\mathfrak{B}}
$$
for all $\vec{x} \in \mathbb{R}^4$. }
\sol Since $T$ is a linear transformation from $\mathbb{R}^4\rightarrow\mathbb{R}^4$, then there exists some $A\in\mathbb{R}^{4\times4}$ so that $[T(\vec{x})]_{\mathfrak{B}}=A[\vec{x}]_{\mathfrak{B}}$. This matrix $A$ can be found by taking $P_{\mathfrak{E}\rightarrow\mathfrak{B}}MP_{\mathfrak{B}\rightarrow\mathfrak{E}}$, where $M$ is the matrix where $T(\vec{x})=M\vec{x}$. So $P_{\mathfrak{B}\rightarrow\mathfrak{E}}$ and $P_{\mathfrak{E}\rightarrow\mathfrak{B}}$ must be found.
$$
P_{\mathfrak{B}\rightarrow\mathfrak{E}} = 
\begin{bmatrix}
2 & 3 & 0 & 0 \\
2 & 1 & 0 & 0 \\
0 & 0 & 1 & 1 \\
0 & 0 & 2 & 1
\end{bmatrix} \text{ and }
P_{\mathfrak{B}\rightarrow\mathfrak{E}} = 
\begin{bmatrix}
2 & 3 & 0 & 0 \\
2 & 1 & 0 & 0 \\
0 & 0 & 1 & 1 \\
0 & 0 & 2 & 1
\end{bmatrix}^{-1}
$$
So then we must find $P_{\mathfrak{B}\rightarrow\mathfrak{E}}$ by inverting $P_{\mathfrak{B}\rightarrow\mathfrak{E}}$.
\begin{align*}
	&\left[\begin{array}{cccc|cccc}
		2 & 3 & 0 & 0 & 1 & 0 & 0 & 0 \\
		2 & 1 & 0 & 0 & 0 & 1 & 0 & 0 \\
		0 & 0 & 1 & 1 & 0 & 0 & 1 & 0 \\
		0 & 0 & 2 & 1 & 0 & 0 & 0 & 1
	\end{array}\right]
	&& \begin{aligned} & R_2 - R_1 \rightarrow R_2 \\ & R_3 - R_4 \rightarrow R_3 \end{aligned}
	&\left[\begin{array}{cccc|cccc}
		2 & 3 & 0 & 0 & 1 & 0 & 0 & 0 \\
		0 & -1 & 0 & 0 & -1 & 1 & 0 & 0 \\
		0 & 0 & -1 & 0 & 0 & 0 & 1 & -1 \\
		0 & 0 & 2 & 1 & 0 & 0 & 0 & 1
	\end{array}\right]
	&& \begin{aligned} & \frac{1}{2}R_1 \rightarrow R_1 \\ & -\frac{1}{2}R_2 \rightarrow R_2 \\ & R_4 + 2R_3 \rightarrow R_3 \end{aligned} \\
	&\left[\begin{array}{cccc|cccc}
		1 & \frac{3}{2} & 0 & 0 & \frac{1}{2} & 0 & 0 & 0 \\
		0 & 1 & 0 & 0 & \frac{1}{2} & -\frac{1}{2} & 0 & 0 \\
		0 & 0 & -1 & 0 & 0 & 0 & 1 & -1 \\
		0 & 0 & 0 & 1 & 0 & 0 & 2 & -1
	\end{array}\right]
	&& \begin{aligned} &R_1 - \frac{3}{2}R_2 \rightarrow R_1 \\ & -R_3 \rightarrow R_3 \end{aligned}
	&\left[\begin{array}{cccc|cccc}
		1 & 0 & 0 & 0 & -\frac{1}{4} & \frac{3}{4} & 0 & 0 \\
		0 & 1 & 0 & 0 & \frac{1}{2} & -\frac{1}{2} & 0 & 0 \\
		0 & 0 & 1 & 0 & 0 & 0 & -1 & 1 \\
		0 & 0 & 0 & 1 & 0 & 0 & 2 & -1
	\end{array}\right]
\end{align*}
So now, we can find A by using the formula from before.
$$
\begin{aligned}
A = P_{\mathfrak{E}\rightarrow\mathfrak{B}}MP_{\mathfrak{B}\rightarrow\mathfrak{E}} & = 
\begin{bmatrix}
-1/4 & 3/4 & 0 & 0 \\
1/2 & -1/2 & 0 & 0 \\
0 & 0 & -1 & 1 \\
0 & 0 & 2 & -1
\end{bmatrix}
\begin{bmatrix}
5 / 2 & 3 / 2 & 0 & 0 \\
-1 & 1 / 2 & 0 & 0 \\
0 & 0 & 5 / 4 & 1 / 4 \\
0 & 0 & -1 / 2 & 1 / 2
\end{bmatrix}
\begin{bmatrix}
2 & 3 & 0 & 0 \\
2 & 1 & 0 & 0 \\
0 & 0 & 1 & 1 \\
0 & 0 & 2 & 1
\end{bmatrix} \\
& =
\begin{bmatrix}
-11/8 & 0 & 0 & 0 \\
7/4 & 1/2 & 0 & 0 \\
0 & 0 & -7/4 & 1/4 \\
0 & 0 & 3 & 0
\end{bmatrix}
\begin{bmatrix}
2 & 3 & 0 & 0 \\
2 & 1 & 0 & 0 \\
0 & 0 & 1 & 1 \\
0 & 0 & 2 & 1
\end{bmatrix} \\
A & =
\begin{bmatrix}
-11/4 & -33/8 & 0 & 0 \\
9/2 & 23/4 & 0 & 0 \\
0 & 0 & -5/4 & -3/2 \\
0 & 0 & 3 & 3
\end{bmatrix}
\end{aligned}
$$
\qs{}{Let $T: \mathbb{P}_3 \rightarrow \mathbb{R}^{2 \times 2}$ be defined by
$$
T\left(a+b x+c x^2\right)=\left[\begin{array}{cc}
b & 2 a \\
2 c & b
\end{array}\right]
$$
and let $\mathfrak{B}$ be the basis for $\mathbb{P}_3$ and $\mathfrak{D}$ the basis for $\mathbb{R}^{2 \times 2}$ :
$$
\mathfrak{B}=\left\langle 1, x, x^2\right\rangle \quad \text { and } \quad \mathfrak{D}=\left\langle\left[\begin{array}{ll}
1 & 0 \\
0 & 0
\end{array}\right],\left[\begin{array}{ll}
0 & 1 \\
0 & 0
\end{array}\right],\left[\begin{array}{ll}
0 & 0 \\
1 & 0
\end{array}\right],\left[\begin{array}{ll}
0 & 0 \\
0 & 1
\end{array}\right]\right\rangle .
$$

Find a matrix $A$ so that
$$
[T(p)]_{\mathfrak{D}}=A[p]_{\mathfrak{B}}
$$
for all $p \in \mathbb{P}_3$. }
\sol Since $T$ is a linear transformation from $\mathbb{P}_3\rightarrow\mathbb{R}^{2\times2}$, and we are given a basis $\mathfrak{B}$ for $\mathbb{P}_3$ and $\mathfrak{D} for \mathbb{R}^{2\times2}$, then there exists some matrix A such that $[T(p)]_{\mathfrak{D}}=A[p]_{\mathfrak{B}}$. The matrix can be constructed, first by computing the transformed elements in the ordered basis $\mathfrak{B}$, see:
$$
\begin{aligned}
T(1) & = \begin{bmatrix} 0 & 0 \\ 2 & 0 \end{bmatrix} \implies \left[\begin{bmatrix} 0 & 0 \\ 2 & 0 \end{bmatrix}\right]_{\mathfrak{D}} = \begin{bmatrix} 0 \\ 0 \\ 2 \\ 0 \end{bmatrix}\\ 
T(x) & = \begin{bmatrix} 1 & 0 \\ 0 & 1 \end{bmatrix} \implies \left[\begin{bmatrix} 1 & 0 \\ 0 & 1 \end{bmatrix}\right]_{\mathfrak{D}} = \begin{bmatrix} 1 \\ 0 \\ 0 \\ 1 \end{bmatrix}\\
T(x^2) & = \begin{bmatrix} 0 & 2 \\ 0 & 0 \end{bmatrix} \implies \left[\begin{bmatrix} 0 & 2 \\ 0 & 0 \end{bmatrix}\right]_{\mathfrak{D}} = \begin{bmatrix} 0 \\ 2 \\ 0 \\ 0 \end{bmatrix}
\end{aligned}
$$
The matrix can now be constructed taking the $\mathfrak{D}$-coordinate vectors that have been calculated as the columns to build a matrix so thus we have
$$
A = \left[
  \begin{array}{cccc}
    \vrule & \vrule & & \vrule\\
    \left[T(b_{1})\right]_{\mathfrak{D}} & \left[T(b_{2})\right]_{\mathfrak{D}} & \ldots & \left[T(b_{n})\right]_{\mathfrak{D}} \\
    \vrule & \vrule & & \vrule 
  \end{array}
\right] =
\begin{bmatrix}
0 & 1 & 0 \\
0 & 0 & 2 \\
2 & 0 & 0 \\
0 & 1 & 0
\end{bmatrix}
$$
\qs{}{
Let $T: \mathbb{R}^3 \rightarrow \mathbb{R}^3$ be given by
$$
T(\vec{x})=\left[\begin{array}{lll}
1 & 0 & 1 \\
1 & 1 & 2 \\
0 & 1 & 1
\end{array}\right] \vec{x}
$$
and $\mathfrak{E}=\left\langle\vec{e}_1, \vec{e}_2, \vec{e}_3\right\rangle$ be the standard basis for $\mathbb{R}^3$. Let
$$
\mathfrak{B}=\left\langle T\left(\vec{e}_1\right), T\left(\vec{e}_2\right), \vec{e}_2\right\rangle
$$

Find a matrix $A \in \mathbb{R}^{3 \times 3}$ so that
$$
[T(\vec{x})]_{\mathfrak{B}}=A[\vec{x}]_{\mathfrak{E}} .
$$
}
\sol To find the matrix $A$, firstly we can substitute some arbitrary vector into the fuction, take $\vec{v} = \begin{bmatrix}v_1 \\ v_2 \\ v_3\end{bmatrix}$. Firstly we must find
$$
T(\vec{v}) = 
\begin{bmatrix}
1 & 0 & 1 \\ 
1 & 1 & 2 \\ 
0 & 1 & 1 
\end{bmatrix}
\begin{bmatrix}
v_1 \\ v_2 \\ v_3
\end{bmatrix} =
\begin{bmatrix}
v_1 + v_3 \\ v_1 + v_2 + 2v_3 \\ v_2 + v_3
\end{bmatrix}
$$
Then when expressing this as the $\mathfrak{B}$-coordinate of the calculated vector, we find that
$$
\left[\begin{bmatrix}
v_1 + v_3 \\ v_1 + v_2 + 2v_3 \\ v_2 + v_3
\end{bmatrix}\right]_{\mathfrak{B}} = \begin{bmatrix} v_1 + v_3 \\ v_2 + v_3 \\ 0 \end{bmatrix}
$$
We can conclude this because of the vectors contained in the orded basis. The result of the linear transformation suggests that we need at least $v_1 + v_3$ of the first element from the ordered basis. Further, we need $v_2 + v_3$ of the second element in the ordered basis because it is the only element which contains a value in the third component of the vector. Since we know what form the $\mathfrak{B}$-coordinates of the output of $T$ will look like, we can generate a matrix.
$$
A\begin{bmatrix}v_1 + v_2 \\ v_2 + v_3 \\ 0\end{bmatrix} \implies A = \begin{bmatrix} 1 & 0 & 1 \\ 0 & 1 & 1 \\ 0 & 0 & 0 \end{bmatrix}
$$
\end{document}
