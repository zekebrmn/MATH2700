\documentclass{report}

%%%%%%%%%%%%%%%%%%%%%%%%%%%%%%%%%
% PACKAGE IMPORTS
%%%%%%%%%%%%%%%%%%%%%%%%%%%%%%%%%


\usepackage[tmargin=2cm,rmargin=1in,lmargin=1in,margin=0.85in,bmargin=2cm,footskip=.2in]{geometry}
\usepackage{amsmath,amsfonts,amsthm,amssymb,mathtools}
\usepackage[varbb]{newpxmath}
\usepackage{xfrac}
\usepackage[makeroom]{cancel}
\usepackage{mathtools}
\usepackage{bookmark}
\usepackage{enumitem}
\usepackage{hyperref,theoremref}
\hypersetup{
	pdftitle={Assignment},
	colorlinks=true, linkcolor=doc!90,
	bookmarksnumbered=true,
	bookmarksopen=true
}
\usepackage[most,many,breakable]{tcolorbox}
\usepackage{xcolor}
\usepackage{varwidth}
\usepackage{varwidth}
\usepackage{etoolbox}
%\usepackage{authblk}
\usepackage{nameref}
\usepackage{multicol,array}
\usepackage{tikz-cd}
\usepackage[ruled,vlined,linesnumbered]{algorithm2e}
\usepackage{comment} % enables the use of multi-line comments (\ifx \fi) 
\usepackage{import}
\usepackage{xifthen}
\usepackage{pdfpages}
\usepackage{transparent}

\newcommand\mycommfont[1]{\footnotesize\ttfamily\textcolor{blue}{#1}}
\SetCommentSty{mycommfont}
\newcommand{\incfig}[1]{%
    \def\svgwidth{\columnwidth}
    \import{./figures/}{#1.pdf_tex}
}

\usepackage{tikzsymbols}
\renewcommand\qedsymbol{$\Laughey$}


%\usepackage{import}
%\usepackage{xifthen}
%\usepackage{pdfpages}
%\usepackage{transparent}


%%%%%%%%%%%%%%%%%%%%%%%%%%%%%%
% SELF MADE COLORS
%%%%%%%%%%%%%%%%%%%%%%%%%%%%%%



\definecolor{myg}{RGB}{56, 140, 70}
\definecolor{myb}{RGB}{45, 111, 177}
\definecolor{myr}{RGB}{199, 68, 64}
\definecolor{mytheorembg}{HTML}{F2F2F9}
\definecolor{mytheoremfr}{HTML}{00007B}
\definecolor{mylenmabg}{HTML}{FFFAF8}
\definecolor{mylenmafr}{HTML}{983b0f}
\definecolor{mypropbg}{HTML}{f2fbfc}
\definecolor{mypropfr}{HTML}{191971}
\definecolor{myexamplebg}{HTML}{F2FBF8}
\definecolor{myexamplefr}{HTML}{88D6D1}
\definecolor{myexampleti}{HTML}{2A7F7F}
\definecolor{mydefinitbg}{HTML}{E5E5FF}
\definecolor{mydefinitfr}{HTML}{3F3FA3}
\definecolor{notesgreen}{RGB}{0,162,0}
\definecolor{myp}{RGB}{197, 92, 212}
\definecolor{mygr}{HTML}{2C3338}
\definecolor{myred}{RGB}{127,0,0}
\definecolor{myyellow}{RGB}{169,121,69}
\definecolor{myexercisebg}{HTML}{F2FBF8}
\definecolor{myexercisefg}{HTML}{88D6D1}


%%%%%%%%%%%%%%%%%%%%%%%%%%%%
% TCOLORBOX SETUPS
%%%%%%%%%%%%%%%%%%%%%%%%%%%%

\setlength{\parindent}{1cm}
%================================
% THEOREM BOX
%================================

\tcbuselibrary{theorems,skins,hooks}
\newtcbtheorem[number within=section]{Theorem}{Theorem}
{%
	enhanced,
	breakable,
	colback = mytheorembg,
	frame hidden,
	boxrule = 0sp,
	borderline west = {2pt}{0pt}{mytheoremfr},
	sharp corners,
	detach title,
	before upper = \tcbtitle\par\smallskip,
	coltitle = mytheoremfr,
	fonttitle = \bfseries\sffamily,
	description font = \mdseries,
	separator sign none,
	segmentation style={solid, mytheoremfr},
}
{th}

\tcbuselibrary{theorems,skins,hooks}
\newtcbtheorem[number within=chapter]{theorem}{Theorem}
{%
	enhanced,
	breakable,
	colback = mytheorembg,
	frame hidden,
	boxrule = 0sp,
	borderline west = {2pt}{0pt}{mytheoremfr},
	sharp corners,
	detach title,
	before upper = \tcbtitle\par\smallskip,
	coltitle = mytheoremfr,
	fonttitle = \bfseries\sffamily,
	description font = \mdseries,
	separator sign none,
	segmentation style={solid, mytheoremfr},
}
{th}


\tcbuselibrary{theorems,skins,hooks}
\newtcolorbox{Theoremcon}
{%
	enhanced
	,breakable
	,colback = mytheorembg
	,frame hidden
	,boxrule = 0sp
	,borderline west = {2pt}{0pt}{mytheoremfr}
	,sharp corners
	,description font = \mdseries
	,separator sign none
}

%================================
% Corollery
%================================
\tcbuselibrary{theorems,skins,hooks}
\newtcbtheorem[number within=section]{Corollary}{Corollary}
{%
	enhanced
	,breakable
	,colback = myp!10
	,frame hidden
	,boxrule = 0sp
	,borderline west = {2pt}{0pt}{myp!85!black}
	,sharp corners
	,detach title
	,before upper = \tcbtitle\par\smallskip
	,coltitle = myp!85!black
	,fonttitle = \bfseries\sffamily
	,description font = \mdseries
	,separator sign none
	,segmentation style={solid, myp!85!black}
}
{th}
\tcbuselibrary{theorems,skins,hooks}
\newtcbtheorem[number within=chapter]{corollary}{Corollary}
{%
	enhanced
	,breakable
	,colback = myp!10
	,frame hidden
	,boxrule = 0sp
	,borderline west = {2pt}{0pt}{myp!85!black}
	,sharp corners
	,detach title
	,before upper = \tcbtitle\par\smallskip
	,coltitle = myp!85!black
	,fonttitle = \bfseries\sffamily
	,description font = \mdseries
	,separator sign none
	,segmentation style={solid, myp!85!black}
}
{th}


%================================
% LENMA
%================================

\tcbuselibrary{theorems,skins,hooks}
\newtcbtheorem[number within=section]{Lenma}{Lenma}
{%
	enhanced,
	breakable,
	colback = mylenmabg,
	frame hidden,
	boxrule = 0sp,
	borderline west = {2pt}{0pt}{mylenmafr},
	sharp corners,
	detach title,
	before upper = \tcbtitle\par\smallskip,
	coltitle = mylenmafr,
	fonttitle = \bfseries\sffamily,
	description font = \mdseries,
	separator sign none,
	segmentation style={solid, mylenmafr},
}
{th}

\tcbuselibrary{theorems,skins,hooks}
\newtcbtheorem[number within=chapter]{lenma}{Lenma}
{%
	enhanced,
	breakable,
	colback = mylenmabg,
	frame hidden,
	boxrule = 0sp,
	borderline west = {2pt}{0pt}{mylenmafr},
	sharp corners,
	detach title,
	before upper = \tcbtitle\par\smallskip,
	coltitle = mylenmafr,
	fonttitle = \bfseries\sffamily,
	description font = \mdseries,
	separator sign none,
	segmentation style={solid, mylenmafr},
}
{th}


%================================
% PROPOSITION
%================================

\tcbuselibrary{theorems,skins,hooks}
\newtcbtheorem[number within=section]{Prop}{Proposition}
{%
	enhanced,
	breakable,
	colback = mypropbg,
	frame hidden,
	boxrule = 0sp,
	borderline west = {2pt}{0pt}{mypropfr},
	sharp corners,
	detach title,
	before upper = \tcbtitle\par\smallskip,
	coltitle = mypropfr,
	fonttitle = \bfseries\sffamily,
	description font = \mdseries,
	separator sign none,
	segmentation style={solid, mypropfr},
}
{th}

\tcbuselibrary{theorems,skins,hooks}
\newtcbtheorem[number within=chapter]{prop}{Proposition}
{%
	enhanced,
	breakable,
	colback = mypropbg,
	frame hidden,
	boxrule = 0sp,
	borderline west = {2pt}{0pt}{mypropfr},
	sharp corners,
	detach title,
	before upper = \tcbtitle\par\smallskip,
	coltitle = mypropfr,
	fonttitle = \bfseries\sffamily,
	description font = \mdseries,
	separator sign none,
	segmentation style={solid, mypropfr},
}
{th}


%================================
% CLAIM
%================================

\tcbuselibrary{theorems,skins,hooks}
\newtcbtheorem[number within=section]{claim}{Claim}
{%
	enhanced
	,breakable
	,colback = myg!10
	,frame hidden
	,boxrule = 0sp
	,borderline west = {2pt}{0pt}{myg}
	,sharp corners
	,detach title
	,before upper = \tcbtitle\par\smallskip
	,coltitle = myg!85!black
	,fonttitle = \bfseries\sffamily
	,description font = \mdseries
	,separator sign none
	,segmentation style={solid, myg!85!black}
}
{th}



%================================
% Exercise
%================================

\tcbuselibrary{theorems,skins,hooks}
\newtcbtheorem[number within=section]{Exercise}{Exercise}
{%
	enhanced,
	breakable,
	colback = myexercisebg,
	frame hidden,
	boxrule = 0sp,
	borderline west = {2pt}{0pt}{myexercisefg},
	sharp corners,
	detach title,
	before upper = \tcbtitle\par\smallskip,
	coltitle = myexercisefg,
	fonttitle = \bfseries\sffamily,
	description font = \mdseries,
	separator sign none,
	segmentation style={solid, myexercisefg},
}
{th}

\tcbuselibrary{theorems,skins,hooks}
\newtcbtheorem[number within=chapter]{exercise}{Exercise}
{%
	enhanced,
	breakable,
	colback = myexercisebg,
	frame hidden,
	boxrule = 0sp,
	borderline west = {2pt}{0pt}{myexercisefg},
	sharp corners,
	detach title,
	before upper = \tcbtitle\par\smallskip,
	coltitle = myexercisefg,
	fonttitle = \bfseries\sffamily,
	description font = \mdseries,
	separator sign none,
	segmentation style={solid, myexercisefg},
}
{th}

%================================
% EXAMPLE BOX
%================================

\newtcbtheorem[number within=section]{Example}{Example}
{%
	colback = myexamplebg
	,breakable
	,colframe = myexamplefr
	,coltitle = myexampleti
	,boxrule = 1pt
	,sharp corners
	,detach title
	,before upper=\tcbtitle\par\smallskip
	,fonttitle = \bfseries
	,description font = \mdseries
	,separator sign none
	,description delimiters parenthesis
}
{ex}

\newtcbtheorem[number within=chapter]{example}{Example}
{%
	colback = myexamplebg
	,breakable
	,colframe = myexamplefr
	,coltitle = myexampleti
	,boxrule = 1pt
	,sharp corners
	,detach title
	,before upper=\tcbtitle\par\smallskip
	,fonttitle = \bfseries
	,description font = \mdseries
	,separator sign none
	,description delimiters parenthesis
}
{ex}

%================================
% DEFINITION BOX
%================================

\newtcbtheorem[number within=section]{Definition}{Definition}{enhanced,
	before skip=2mm,after skip=2mm, colback=red!5,colframe=red!80!black,boxrule=0.5mm,
	attach boxed title to top left={xshift=1cm,yshift*=1mm-\tcboxedtitleheight}, varwidth boxed title*=-3cm,
	boxed title style={frame code={
					\path[fill=tcbcolback]
					([yshift=-1mm,xshift=-1mm]frame.north west)
					arc[start angle=0,end angle=180,radius=1mm]
					([yshift=-1mm,xshift=1mm]frame.north east)
					arc[start angle=180,end angle=0,radius=1mm];
					\path[left color=tcbcolback!60!black,right color=tcbcolback!60!black,
						middle color=tcbcolback!80!black]
					([xshift=-2mm]frame.north west) -- ([xshift=2mm]frame.north east)
					[rounded corners=1mm]-- ([xshift=1mm,yshift=-1mm]frame.north east)
					-- (frame.south east) -- (frame.south west)
					-- ([xshift=-1mm,yshift=-1mm]frame.north west)
					[sharp corners]-- cycle;
				},interior engine=empty,
		},
	fonttitle=\bfseries,
	title={#2},#1}{def}
\newtcbtheorem[number within=chapter]{definition}{Definition}{enhanced,
	before skip=2mm,after skip=2mm, colback=red!5,colframe=red!80!black,boxrule=0.5mm,
	attach boxed title to top left={xshift=1cm,yshift*=1mm-\tcboxedtitleheight}, varwidth boxed title*=-3cm,
	boxed title style={frame code={
					\path[fill=tcbcolback]
					([yshift=-1mm,xshift=-1mm]frame.north west)
					arc[start angle=0,end angle=180,radius=1mm]
					([yshift=-1mm,xshift=1mm]frame.north east)
					arc[start angle=180,end angle=0,radius=1mm];
					\path[left color=tcbcolback!60!black,right color=tcbcolback!60!black,
						middle color=tcbcolback!80!black]
					([xshift=-2mm]frame.north west) -- ([xshift=2mm]frame.north east)
					[rounded corners=1mm]-- ([xshift=1mm,yshift=-1mm]frame.north east)
					-- (frame.south east) -- (frame.south west)
					-- ([xshift=-1mm,yshift=-1mm]frame.north west)
					[sharp corners]-- cycle;
				},interior engine=empty,
		},
	fonttitle=\bfseries,
	title={#2},#1}{def}



%================================
% Solution BOX
%================================

\makeatletter
\newtcbtheorem{question}{Question}{enhanced,
	breakable,
	colback=white,
	colframe=myb!80!black,
	attach boxed title to top left={yshift*=-\tcboxedtitleheight},
	fonttitle=\bfseries,
	title={#2},
	boxed title size=title,
	boxed title style={%
			sharp corners,
			rounded corners=northwest,
			colback=tcbcolframe,
			boxrule=0pt,
		},
	underlay boxed title={%
			\path[fill=tcbcolframe] (title.south west)--(title.south east)
			to[out=0, in=180] ([xshift=5mm]title.east)--
			(title.center-|frame.east)
			[rounded corners=\kvtcb@arc] |-
			(frame.north) -| cycle;
		},
	#1
}{def}
\makeatother

%================================
% SOLUTION BOX
%================================

\makeatletter
\newtcolorbox{solution}{enhanced,
	breakable,
	colback=white,
	colframe=myg!80!black,
	attach boxed title to top left={yshift*=-\tcboxedtitleheight},
	title=Solution,
	boxed title size=title,
	boxed title style={%
			sharp corners,
			rounded corners=northwest,
			colback=tcbcolframe,
			boxrule=0pt,
		},
	underlay boxed title={%
			\path[fill=tcbcolframe] (title.south west)--(title.south east)
			to[out=0, in=180] ([xshift=5mm]title.east)--
			(title.center-|frame.east)
			[rounded corners=\kvtcb@arc] |-
			(frame.north) -| cycle;
		},
}
\makeatother

%================================
% Question BOX
%================================

\makeatletter
\newtcbtheorem{qstion}{Question}{enhanced,
	breakable,
	colback=white,
	colframe=mygr,
	attach boxed title to top left={yshift*=-\tcboxedtitleheight},
	fonttitle=\bfseries,
	title={#2},
	boxed title size=title,
	boxed title style={%
			sharp corners,
			rounded corners=northwest,
			colback=tcbcolframe,
			boxrule=0pt,
		},
	underlay boxed title={%
			\path[fill=tcbcolframe] (title.south west)--(title.south east)
			to[out=0, in=180] ([xshift=5mm]title.east)--
			(title.center-|frame.east)
			[rounded corners=\kvtcb@arc] |-
			(frame.north) -| cycle;
		},
	#1
}{def}
\makeatother

\newtcbtheorem[number within=chapter]{wconc}{Wrong Concept}{
	breakable,
	enhanced,
	colback=white,
	colframe=myr,
	arc=0pt,
	outer arc=0pt,
	fonttitle=\bfseries\sffamily\large,
	colbacktitle=myr,
	attach boxed title to top left={},
	boxed title style={
			enhanced,
			skin=enhancedfirst jigsaw,
			arc=3pt,
			bottom=0pt,
			interior style={fill=myr}
		},
	#1
}{def}



%================================
% NOTE BOX
%================================

\usetikzlibrary{arrows,calc,shadows.blur}
\tcbuselibrary{skins}
\newtcolorbox{note}[1][]{%
	enhanced jigsaw,
	colback=gray!20!white,%
	colframe=gray!80!black,
	size=small,
	boxrule=1pt,
	title=\textbf{Note:-},
	halign title=flush center,
	coltitle=black,
	breakable,
	drop shadow=black!50!white,
	attach boxed title to top left={xshift=1cm,yshift=-\tcboxedtitleheight/2,yshifttext=-\tcboxedtitleheight/2},
	minipage boxed title=1.5cm,
	boxed title style={%
			colback=white,
			size=fbox,
			boxrule=1pt,
			boxsep=2pt,
			underlay={%
					\coordinate (dotA) at ($(interior.west) + (-0.5pt,0)$);
					\coordinate (dotB) at ($(interior.east) + (0.5pt,0)$);
					\begin{scope}
						\clip (interior.north west) rectangle ([xshift=3ex]interior.east);
						\filldraw [white, blur shadow={shadow opacity=60, shadow yshift=-.75ex}, rounded corners=2pt] (interior.north west) rectangle (interior.south east);
					\end{scope}
					\begin{scope}[gray!80!black]
						\fill (dotA) circle (2pt);
						\fill (dotB) circle (2pt);
					\end{scope}
				},
		},
	#1,
}

%%%%%%%%%%%%%%%%%%%%%%%%%%%%%%
% SELF MADE COMMANDS
%%%%%%%%%%%%%%%%%%%%%%%%%%%%%%


\newcommand{\thm}[2]{\begin{Theorem}{#1}{}#2\end{Theorem}}
\newcommand{\cor}[2]{\begin{Corollary}{#1}{}#2\end{Corollary}}
\newcommand{\mlenma}[2]{\begin{Lenma}{#1}{}#2\end{Lenma}}
\newcommand{\mprop}[2]{\begin{Prop}{#1}{}#2\end{Prop}}
\newcommand{\clm}[3]{\begin{claim}{#1}{#2}#3\end{claim}}
\newcommand{\wc}[2]{\begin{wconc}{#1}{}\setlength{\parindent}{1cm}#2\end{wconc}}
\newcommand{\thmcon}[1]{\begin{Theoremcon}{#1}\end{Theoremcon}}
\newcommand{\ex}[2]{\begin{Example}{#1}{}#2\end{Example}}
\newcommand{\dfn}[2]{\begin{Definition}[colbacktitle=red!75!black]{#1}{}#2\end{Definition}}
\newcommand{\dfnc}[2]{\begin{definition}[colbacktitle=red!75!black]{#1}{}#2\end{definition}}
\newcommand{\qs}[2]{\begin{question}{#1}{}#2\end{question}}
\newcommand{\pf}[2]{\begin{myproof}[#1]#2\end{myproof}}
\newcommand{\nt}[1]{\begin{note}#1\end{note}}

\newcommand*\circled[1]{\tikz[baseline=(char.base)]{
		\node[shape=circle,draw,inner sep=1pt] (char) {#1};}}
\newcommand\getcurrentref[1]{%
	\ifnumequal{\value{#1}}{0}
	{??}
	{\the\value{#1}}%
}
\newcommand{\getCurrentSectionNumber}{\getcurrentref{section}}
\newenvironment{myproof}[1][\proofname]{%
	\proof[\bfseries #1: ]%
}{\endproof}

\newcommand{\mclm}[2]{\begin{myclaim}[#1]#2\end{myclaim}}
\newenvironment{myclaim}[1][\claimname]{\proof[\bfseries #1: ]}{}

\newcounter{mylabelcounter}

\makeatletter
\newcommand{\setword}[2]{%
	\phantomsection
	#1\def\@currentlabel{\unexpanded{#1}}\label{#2}%
}
\makeatother




\tikzset{
	symbol/.style={
			draw=none,
			every to/.append style={
					edge node={node [sloped, allow upside down, auto=false]{$#1$}}}
		}
}


% deliminators
\DeclarePairedDelimiter{\abs}{\lvert}{\rvert}
\DeclarePairedDelimiter{\norm}{\lVert}{\rVert}

\DeclarePairedDelimiter{\ceil}{\lceil}{\rceil}
\DeclarePairedDelimiter{\floor}{\lfloor}{\rfloor}
\DeclarePairedDelimiter{\round}{\lfloor}{\rceil}

\newsavebox\diffdbox
\newcommand{\slantedromand}{{\mathpalette\makesl{d}}}
\newcommand{\makesl}[2]{%
\begingroup
\sbox{\diffdbox}{$\mathsurround=0pt#1\mathrm{#2}$}%
\pdfsave
\pdfsetmatrix{1 0 0.2 1}%
\rlap{\usebox{\diffdbox}}%
\pdfrestore
\hskip\wd\diffdbox
\endgroup
}
\newcommand{\dd}[1][]{\ensuremath{\mathop{}\!\ifstrempty{#1}{%
\slantedromand\@ifnextchar^{\hspace{0.2ex}}{\hspace{0.1ex}}}%
{\slantedromand\hspace{0.2ex}^{#1}}}}
\ProvideDocumentCommand\dv{o m g}{%
  \ensuremath{%
    \IfValueTF{#3}{%
      \IfNoValueTF{#1}{%
        \frac{\dd #2}{\dd #3}%
      }{%
        \frac{\dd^{#1} #2}{\dd #3^{#1}}%
      }%
    }{%
      \IfNoValueTF{#1}{%
        \frac{\dd}{\dd #2}%
      }{%
        \frac{\dd^{#1}}{\dd #2^{#1}}%
      }%
    }%
  }%
}
\providecommand*{\pdv}[3][]{\frac{\partial^{#1}#2}{\partial#3^{#1}}}
%  - others
\DeclareMathOperator{\Lap}{\mathcal{L}}
\DeclareMathOperator{\Var}{Var} % varience
\DeclareMathOperator{\Cov}{Cov} % covarience
\DeclareMathOperator{\E}{E} % expected

% Since the amsthm package isn't loaded

% I prefer the slanted \leq
\let\oldleq\leq % save them in case they're every wanted
\let\oldgeq\geq
\renewcommand{\leq}{\leqslant}
\renewcommand{\geq}{\geqslant}

% % redefine matrix env to allow for alignment, use r as default
% \renewcommand*\env@matrix[1][r]{\hskip -\arraycolsep
%     \let\@ifnextchar\new@ifnextchar
%     \array{*\c@MaxMatrixCols #1}}


%\usepackage{framed}
%\usepackage{titletoc}
%\usepackage{etoolbox}
%\usepackage{lmodern}


%\patchcmd{\tableofcontents}{\contentsname}{\sffamily\contentsname}{}{}

%\renewenvironment{leftbar}
%{\def\FrameCommand{\hspace{6em}%
%		{\color{myyellow}\vrule width 2pt depth 6pt}\hspace{1em}}%
%	\MakeFramed{\parshape 1 0cm \dimexpr\textwidth-6em\relax\FrameRestore}\vskip2pt%
%}
%{\endMakeFramed}

%\titlecontents{chapter}
%[0em]{\vspace*{2\baselineskip}}
%{\parbox{4.5em}{%
%		\hfill\Huge\sffamily\bfseries\color{myred}\thecontentspage}%
%	\vspace*{-2.3\baselineskip}\leftbar\textsc{\small\chaptername~\thecontentslabel}\\\sffamily}
%{}{\endleftbar}
%\titlecontents{section}
%[8.4em]
%{\sffamily\contentslabel{3em}}{}{}
%{\hspace{0.5em}\nobreak\itshape\color{myred}\contentspage}
%\titlecontents{subsection}
%[8.4em]
%{\sffamily\contentslabel{3em}}{}{}  
%{\hspace{0.5em}\nobreak\itshape\color{myred}\contentspage}



%%%%%%%%%%%%%%%%%%%%%%%%%%%%%%%%%%%%%%%%%%%
% TABLE OF CONTENTS
%%%%%%%%%%%%%%%%%%%%%%%%%%%%%%%%%%%%%%%%%%%

\usepackage{tikz}
\definecolor{doc}{RGB}{0,60,110}
\usepackage{titletoc}
\contentsmargin{0cm}
\titlecontents{chapter}[3.7pc]
{\addvspace{30pt}%
	\begin{tikzpicture}[remember picture, overlay]%
		\draw[fill=doc!60,draw=doc!60] (-7,-.1) rectangle (-0.9,.5);%
		\pgftext[left,x=-3.5cm,y=0.2cm]{\color{white}\Large\sc\bfseries Chapter\ \thecontentslabel};%
	\end{tikzpicture}\color{doc!60}\large\sc\bfseries}%
{}
{}
{\;\titlerule\;\large\sc\bfseries Page \thecontentspage
	\begin{tikzpicture}[remember picture, overlay]
		\draw[fill=doc!60,draw=doc!60] (2pt,0) rectangle (4,0.1pt);
	\end{tikzpicture}}%
\titlecontents{section}[3.7pc]
{\addvspace{2pt}}
{\contentslabel[\thecontentslabel]{2pc}}
{}
{\hfill\small \thecontentspage}
[]
\titlecontents*{subsection}[3.7pc]
{\addvspace{-1pt}\small}
{}
{}
{\ --- \small\thecontentspage}
[ \textbullet\ ][]

\makeatletter
\renewcommand{\tableofcontents}{%
	\chapter*{%
	  \vspace*{-20\p@}%
	  \begin{tikzpicture}[remember picture, overlay]%
		  \pgftext[right,x=15cm,y=0.2cm]{\color{doc!60}\Huge\sc\bfseries \contentsname};%
		  \draw[fill=doc!60,draw=doc!60] (13,-.75) rectangle (20,1);%
		  \clip (13,-.75) rectangle (20,1);
		  \pgftext[right,x=15cm,y=0.2cm]{\color{white}\Huge\sc\bfseries \contentsname};%
	  \end{tikzpicture}}%
	\@starttoc{toc}}
\makeatother


%From M275 "Topology" at SJSU
\newcommand{\id}{\mathrm{id}}
\newcommand{\taking}[1]{\xrightarrow{#1}}
\newcommand{\inv}{^{-1}}

%From M170 "Introduction to Graph Theory" at SJSU
\DeclareMathOperator{\diam}{diam}
\DeclareMathOperator{\ord}{ord}
\newcommand{\defeq}{\overset{\mathrm{def}}{=}}

%From the USAMO .tex files
\newcommand{\ts}{\textsuperscript}
\newcommand{\dg}{^\circ}
\newcommand{\ii}{\item}

% % From Math 55 and Math 145 at Harvard
% \newenvironment{subproof}[1][Proof]{%
% \begin{proof}[#1] \renewcommand{\qedsymbol}{$\blacksquare$}}%
% {\end{proof}}

\newcommand{\liff}{\leftrightarrow}
\newcommand{\lthen}{\rightarrow}
\newcommand{\opname}{\operatorname}
\newcommand{\surjto}{\twoheadrightarrow}
\newcommand{\injto}{\hookrightarrow}
\newcommand{\On}{\mathrm{On}} % ordinals
\DeclareMathOperator{\img}{im} % Image
\DeclareMathOperator{\Img}{Im} % Image
\DeclareMathOperator{\coker}{coker} % Cokernel
\DeclareMathOperator{\Coker}{Coker} % Cokernel
\DeclareMathOperator{\Ker}{Ker} % Kernel
\DeclareMathOperator{\rank}{rank}
\DeclareMathOperator{\Spec}{Spec} % spectrum
\DeclareMathOperator{\Tr}{Tr} % trace
\DeclareMathOperator{\pr}{pr} % projection
\DeclareMathOperator{\ext}{ext} % extension
\DeclareMathOperator{\pred}{pred} % predecessor
\DeclareMathOperator{\dom}{dom} % domain
\DeclareMathOperator{\ran}{ran} % range
\DeclareMathOperator{\Hom}{Hom} % homomorphism
\DeclareMathOperator{\Mor}{Mor} % morphisms
\DeclareMathOperator{\End}{End} % endomorphism

\newcommand{\eps}{\epsilon}
\newcommand{\veps}{\varepsilon}
\newcommand{\ol}{\overline}
\newcommand{\ul}{\underline}
\newcommand{\wt}{\widetilde}
\newcommand{\wh}{\widehat}
\newcommand{\vocab}[1]{\textbf{\color{blue} #1}}
\providecommand{\half}{\frac{1}{2}}
\newcommand{\dang}{\measuredangle} %% Directed angle
\newcommand{\ray}[1]{\overrightarrow{#1}}
\newcommand{\seg}[1]{\overline{#1}}
\newcommand{\arc}[1]{\wideparen{#1}}
\DeclareMathOperator{\cis}{cis}
\DeclareMathOperator*{\lcm}{lcm}
\DeclareMathOperator*{\argmin}{arg min}
\DeclareMathOperator*{\argmax}{arg max}
\newcommand{\cycsum}{\sum_{\mathrm{cyc}}}
\newcommand{\symsum}{\sum_{\mathrm{sym}}}
\newcommand{\cycprod}{\prod_{\mathrm{cyc}}}
\newcommand{\symprod}{\prod_{\mathrm{sym}}}
\newcommand{\Qed}{\begin{flushright}\qed\end{flushright}}
\newcommand{\parinn}{\setlength{\parindent}{1cm}}
\newcommand{\parinf}{\setlength{\parindent}{0cm}}
% \newcommand{\norm}{\|\cdot\|}
\newcommand{\inorm}{\norm_{\infty}}
\newcommand{\opensets}{\{V_{\alpha}\}_{\alpha\in I}}
\newcommand{\oset}{V_{\alpha}}
\newcommand{\opset}[1]{V_{\alpha_{#1}}}
\newcommand{\lub}{\text{lub}}
\newcommand{\del}[2]{\frac{\partial #1}{\partial #2}}
\newcommand{\Del}[3]{\frac{\partial^{#1} #2}{\partial^{#1} #3}}
\newcommand{\deld}[2]{\dfrac{\partial #1}{\partial #2}}
\newcommand{\Deld}[3]{\dfrac{\partial^{#1} #2}{\partial^{#1} #3}}
\newcommand{\lm}{\lambda}
\newcommand{\uin}{\mathbin{\rotatebox[origin=c]{90}{$\in$}}}
\newcommand{\usubset}{\mathbin{\rotatebox[origin=c]{90}{$\subset$}}}
\newcommand{\lt}{\left}
\newcommand{\rt}{\right}
\newcommand{\bs}[1]{\boldsymbol{#1}}
\newcommand{\exs}{\exists}
\newcommand{\st}{\strut}
\newcommand{\dps}[1]{\displaystyle{#1}}

\newcommand{\sol}{\setlength{\parindent}{0cm}\textbf{\textit{Solution:}}\setlength{\parindent}{1cm} }
\newcommand{\solve}[1]{\setlength{\parindent}{0cm}\textbf{\textit{Solution: }}\setlength{\parindent}{1cm}#1 \Qed}

% Things Lie
\newcommand{\kb}{\mathfrak b}
\newcommand{\kg}{\mathfrak g}
\newcommand{\kh}{\mathfrak h}
\newcommand{\kn}{\mathfrak n}
\newcommand{\ku}{\mathfrak u}
\newcommand{\kz}{\mathfrak z}
\DeclareMathOperator{\Ext}{Ext} % Ext functor
\DeclareMathOperator{\Tor}{Tor} % Tor functor
\newcommand{\gl}{\opname{\mathfrak{gl}}} % frak gl group
\renewcommand{\sl}{\opname{\mathfrak{sl}}} % frak sl group chktex 6

% More script letters etc.
\newcommand{\SA}{\mathcal A}
\newcommand{\SB}{\mathcal B}
\newcommand{\SC}{\mathcal C}
\newcommand{\SF}{\mathcal F}
\newcommand{\SG}{\mathcal G}
\newcommand{\SH}{\mathcal H}
\newcommand{\OO}{\mathcal O}

\newcommand{\SCA}{\mathscr A}
\newcommand{\SCB}{\mathscr B}
\newcommand{\SCC}{\mathscr C}
\newcommand{\SCD}{\mathscr D}
\newcommand{\SCE}{\mathscr E}
\newcommand{\SCF}{\mathscr F}
\newcommand{\SCG}{\mathscr G}
\newcommand{\SCH}{\mathscr H}

% Mathfrak primes
\newcommand{\km}{\mathfrak m}
\newcommand{\kp}{\mathfrak p}
\newcommand{\kq}{\mathfrak q}

% number sets
\newcommand{\RR}[1][]{\ensuremath{\ifstrempty{#1}{\mathbb{R}}{\mathbb{R}^{#1}}}}
\newcommand{\NN}[1][]{\ensuremath{\ifstrempty{#1}{\mathbb{N}}{\mathbb{N}^{#1}}}}
\newcommand{\ZZ}[1][]{\ensuremath{\ifstrempty{#1}{\mathbb{Z}}{\mathbb{Z}^{#1}}}}
\newcommand{\QQ}[1][]{\ensuremath{\ifstrempty{#1}{\mathbb{Q}}{\mathbb{Q}^{#1}}}}
\newcommand{\CC}[1][]{\ensuremath{\ifstrempty{#1}{\mathbb{C}}{\mathbb{C}^{#1}}}}
\newcommand{\PP}[1][]{\ensuremath{\ifstrempty{#1}{\mathbb{P}}{\mathbb{P}^{#1}}}}
\newcommand{\HH}[1][]{\ensuremath{\ifstrempty{#1}{\mathbb{H}}{\mathbb{H}^{#1}}}}
\newcommand{\FF}[1][]{\ensuremath{\ifstrempty{#1}{\mathbb{F}}{\mathbb{F}^{#1}}}}
% expected value
\newcommand{\EE}{\ensuremath{\mathbb{E}}}
\newcommand{\charin}{\text{ char }}
\DeclareMathOperator{\sign}{sign}
\DeclareMathOperator{\Aut}{Aut}
\DeclareMathOperator{\Inn}{Inn}
\DeclareMathOperator{\Syl}{Syl}
\DeclareMathOperator{\Gal}{Gal}
\DeclareMathOperator{\GL}{GL} % General linear group
\DeclareMathOperator{\SL}{SL} % Special linear group

%---------------------------------------
% BlackBoard Math Fonts :-
%---------------------------------------

%Captital Letters
\newcommand{\bbA}{\mathbb{A}}	\newcommand{\bbB}{\mathbb{B}}
\newcommand{\bbC}{\mathbb{C}}	\newcommand{\bbD}{\mathbb{D}}
\newcommand{\bbE}{\mathbb{E}}	\newcommand{\bbF}{\mathbb{F}}
\newcommand{\bbG}{\mathbb{G}}	\newcommand{\bbH}{\mathbb{H}}
\newcommand{\bbI}{\mathbb{I}}	\newcommand{\bbJ}{\mathbb{J}}
\newcommand{\bbK}{\mathbb{K}}	\newcommand{\bbL}{\mathbb{L}}
\newcommand{\bbM}{\mathbb{M}}	\newcommand{\bbN}{\mathbb{N}}
\newcommand{\bbO}{\mathbb{O}}	\newcommand{\bbP}{\mathbb{P}}
\newcommand{\bbQ}{\mathbb{Q}}	\newcommand{\bbR}{\mathbb{R}}
\newcommand{\bbS}{\mathbb{S}}	\newcommand{\bbT}{\mathbb{T}}
\newcommand{\bbU}{\mathbb{U}}	\newcommand{\bbV}{\mathbb{V}}
\newcommand{\bbW}{\mathbb{W}}	\newcommand{\bbX}{\mathbb{X}}
\newcommand{\bbY}{\mathbb{Y}}	\newcommand{\bbZ}{\mathbb{Z}}

%---------------------------------------
% MathCal Fonts :-
%---------------------------------------

%Captital Letters
\newcommand{\mcA}{\mathcal{A}}	\newcommand{\mcB}{\mathcal{B}}
\newcommand{\mcC}{\mathcal{C}}	\newcommand{\mcD}{\mathcal{D}}
\newcommand{\mcE}{\mathcal{E}}	\newcommand{\mcF}{\mathcal{F}}
\newcommand{\mcG}{\mathcal{G}}	\newcommand{\mcH}{\mathcal{H}}
\newcommand{\mcI}{\mathcal{I}}	\newcommand{\mcJ}{\mathcal{J}}
\newcommand{\mcK}{\mathcal{K}}	\newcommand{\mcL}{\mathcal{L}}
\newcommand{\mcM}{\mathcal{M}}	\newcommand{\mcN}{\mathcal{N}}
\newcommand{\mcO}{\mathcal{O}}	\newcommand{\mcP}{\mathcal{P}}
\newcommand{\mcQ}{\mathcal{Q}}	\newcommand{\mcR}{\mathcal{R}}
\newcommand{\mcS}{\mathcal{S}}	\newcommand{\mcT}{\mathcal{T}}
\newcommand{\mcU}{\mathcal{U}}	\newcommand{\mcV}{\mathcal{V}}
\newcommand{\mcW}{\mathcal{W}}	\newcommand{\mcX}{\mathcal{X}}
\newcommand{\mcY}{\mathcal{Y}}	\newcommand{\mcZ}{\mathcal{Z}}


%---------------------------------------
% Bold Math Fonts :-
%---------------------------------------

%Captital Letters
\newcommand{\bmA}{\boldsymbol{A}}	\newcommand{\bmB}{\boldsymbol{B}}
\newcommand{\bmC}{\boldsymbol{C}}	\newcommand{\bmD}{\boldsymbol{D}}
\newcommand{\bmE}{\boldsymbol{E}}	\newcommand{\bmF}{\boldsymbol{F}}
\newcommand{\bmG}{\boldsymbol{G}}	\newcommand{\bmH}{\boldsymbol{H}}
\newcommand{\bmI}{\boldsymbol{I}}	\newcommand{\bmJ}{\boldsymbol{J}}
\newcommand{\bmK}{\boldsymbol{K}}	\newcommand{\bmL}{\boldsymbol{L}}
\newcommand{\bmM}{\boldsymbol{M}}	\newcommand{\bmN}{\boldsymbol{N}}
\newcommand{\bmO}{\boldsymbol{O}}	\newcommand{\bmP}{\boldsymbol{P}}
\newcommand{\bmQ}{\boldsymbol{Q}}	\newcommand{\bmR}{\boldsymbol{R}}
\newcommand{\bmS}{\boldsymbol{S}}	\newcommand{\bmT}{\boldsymbol{T}}
\newcommand{\bmU}{\boldsymbol{U}}	\newcommand{\bmV}{\boldsymbol{V}}
\newcommand{\bmW}{\boldsymbol{W}}	\newcommand{\bmX}{\boldsymbol{X}}
\newcommand{\bmY}{\boldsymbol{Y}}	\newcommand{\bmZ}{\boldsymbol{Z}}
%Small Letters
\newcommand{\bma}{\boldsymbol{a}}	\newcommand{\bmb}{\boldsymbol{b}}
\newcommand{\bmc}{\boldsymbol{c}}	\newcommand{\bmd}{\boldsymbol{d}}
\newcommand{\bme}{\boldsymbol{e}}	\newcommand{\bmf}{\boldsymbol{f}}
\newcommand{\bmg}{\boldsymbol{g}}	\newcommand{\bmh}{\boldsymbol{h}}
\newcommand{\bmi}{\boldsymbol{i}}	\newcommand{\bmj}{\boldsymbol{j}}
\newcommand{\bmk}{\boldsymbol{k}}	\newcommand{\bml}{\boldsymbol{l}}
\newcommand{\bmm}{\boldsymbol{m}}	\newcommand{\bmn}{\boldsymbol{n}}
\newcommand{\bmo}{\boldsymbol{o}}	\newcommand{\bmp}{\boldsymbol{p}}
\newcommand{\bmq}{\boldsymbol{q}}	\newcommand{\bmr}{\boldsymbol{r}}
\newcommand{\bms}{\boldsymbol{s}}	\newcommand{\bmt}{\boldsymbol{t}}
\newcommand{\bmu}{\boldsymbol{u}}	\newcommand{\bmv}{\boldsymbol{v}}
\newcommand{\bmw}{\boldsymbol{w}}	\newcommand{\bmx}{\boldsymbol{x}}
\newcommand{\bmy}{\boldsymbol{y}}	\newcommand{\bmz}{\boldsymbol{z}}

%---------------------------------------
% Scr Math Fonts :-
%---------------------------------------

\newcommand{\sA}{{\mathscr{A}}}   \newcommand{\sB}{{\mathscr{B}}}
\newcommand{\sC}{{\mathscr{C}}}   \newcommand{\sD}{{\mathscr{D}}}
\newcommand{\sE}{{\mathscr{E}}}   \newcommand{\sF}{{\mathscr{F}}}
\newcommand{\sG}{{\mathscr{G}}}   \newcommand{\sH}{{\mathscr{H}}}
\newcommand{\sI}{{\mathscr{I}}}   \newcommand{\sJ}{{\mathscr{J}}}
\newcommand{\sK}{{\mathscr{K}}}   \newcommand{\sL}{{\mathscr{L}}}
\newcommand{\sM}{{\mathscr{M}}}   \newcommand{\sN}{{\mathscr{N}}}
\newcommand{\sO}{{\mathscr{O}}}   \newcommand{\sP}{{\mathscr{P}}}
\newcommand{\sQ}{{\mathscr{Q}}}   \newcommand{\sR}{{\mathscr{R}}}
\newcommand{\sS}{{\mathscr{S}}}   \newcommand{\sT}{{\mathscr{T}}}
\newcommand{\sU}{{\mathscr{U}}}   \newcommand{\sV}{{\mathscr{V}}}
\newcommand{\sW}{{\mathscr{W}}}   \newcommand{\sX}{{\mathscr{X}}}
\newcommand{\sY}{{\mathscr{Y}}}   \newcommand{\sZ}{{\mathscr{Z}}}


%---------------------------------------
% Math Fraktur Font
%---------------------------------------

%Captital Letters
\newcommand{\mfA}{\mathfrak{A}}	\newcommand{\mfB}{\mathfrak{B}}
\newcommand{\mfC}{\mathfrak{C}}	\newcommand{\mfD}{\mathfrak{D}}
\newcommand{\mfE}{\mathfrak{E}}	\newcommand{\mfF}{\mathfrak{F}}
\newcommand{\mfG}{\mathfrak{G}}	\newcommand{\mfH}{\mathfrak{H}}
\newcommand{\mfI}{\mathfrak{I}}	\newcommand{\mfJ}{\mathfrak{J}}
\newcommand{\mfK}{\mathfrak{K}}	\newcommand{\mfL}{\mathfrak{L}}
\newcommand{\mfM}{\mathfrak{M}}	\newcommand{\mfN}{\mathfrak{N}}
\newcommand{\mfO}{\mathfrak{O}}	\newcommand{\mfP}{\mathfrak{P}}
\newcommand{\mfQ}{\mathfrak{Q}}	\newcommand{\mfR}{\mathfrak{R}}
\newcommand{\mfS}{\mathfrak{S}}	\newcommand{\mfT}{\mathfrak{T}}
\newcommand{\mfU}{\mathfrak{U}}	\newcommand{\mfV}{\mathfrak{V}}
\newcommand{\mfW}{\mathfrak{W}}	\newcommand{\mfX}{\mathfrak{X}}
\newcommand{\mfY}{\mathfrak{Y}}	\newcommand{\mfZ}{\mathfrak{Z}}
%Small Letters
\newcommand{\mfa}{\mathfrak{a}}	\newcommand{\mfb}{\mathfrak{b}}
\newcommand{\mfc}{\mathfrak{c}}	\newcommand{\mfd}{\mathfrak{d}}
\newcommand{\mfe}{\mathfrak{e}}	\newcommand{\mff}{\mathfrak{f}}
\newcommand{\mfg}{\mathfrak{g}}	\newcommand{\mfh}{\mathfrak{h}}
\newcommand{\mfi}{\mathfrak{i}}	\newcommand{\mfj}{\mathfrak{j}}
\newcommand{\mfk}{\mathfrak{k}}	\newcommand{\mfl}{\mathfrak{l}}
\newcommand{\mfm}{\mathfrak{m}}	\newcommand{\mfn}{\mathfrak{n}}
\newcommand{\mfo}{\mathfrak{o}}	\newcommand{\mfp}{\mathfrak{p}}
\newcommand{\mfq}{\mathfrak{q}}	\newcommand{\mfr}{\mathfrak{r}}
\newcommand{\mfs}{\mathfrak{s}}	\newcommand{\mft}{\mathfrak{t}}
\newcommand{\mfu}{\mathfrak{u}}	\newcommand{\mfv}{\mathfrak{v}}
\newcommand{\mfw}{\mathfrak{w}}	\newcommand{\mfx}{\mathfrak{x}}
\newcommand{\mfy}{\mathfrak{y}}	\newcommand{\mfz}{\mathfrak{z}}


\title{\Huge{Math 2700.009}\\Problem Set 10}
\author{\huge{Ezekiel Berumen}}
\date{02 April 2024}

\begin{document}

\maketitle
\newpage

\qs{}{
Compute the determinant of the following by doing a cofactor expansion about
any row or column of your choosing.
$$\begin{bmatrix}
1&2&3&4&5\\
0&-1&1&3&2\\
0&0&0&0&1\\
1&6&-1&2&4\\
1&-1&1&0&1 
\end{bmatrix}$$
}
\begin{note}
Computing the determinant of a matrix by computing the cofactor expansion about a given row or column will always product the same result regardless of which column or row is picked.  In this case it is a good idea to select the row or column with the most zeros,  since it means a majority of the summands will be nullified.
\end{note}
\sol For this cofactor expansion,  we can iterate about $i=3$.  \\
$$
\operatorname{det}
\begin{bmatrix}
1&2&3&4&5\\
0&-1&1&3&2\\
0&0&0&0&1\\
1&6&-1&2&4\\
1&-1&1&0&1 
\end{bmatrix}
$$
$$
\begin{aligned}
& =
0\operatorname{det}
\begin{bmatrix}
2&3&4&5\\
-1&1&3&2\\
6&-1&2&4\\
-1&1&0&1 
\end{bmatrix} - 0\operatorname{det}
\begin{bmatrix}
1&3&4&5\\
0&1&3&2\\
1&-1&2&4\\
1&1&0&1 
\end{bmatrix} +0\operatorname{det}
\begin{bmatrix}
1&2&4&5\\
0&-1&3&2\\
1&6&2&4\\
1&-1&0&1 
\end{bmatrix} -0\operatorname{det}
\begin{bmatrix}
1&2&3&5\\
0&-1&1&2\\
1&6&-1&4\\
1&-1&1&1 
\end{bmatrix} + 1\operatorname{det}
\begin{bmatrix}
1&2&3&4\\
0&-1&1&3\\
1&6&-1&2\\
1&-1&1&0 
\end{bmatrix} \\
& = \operatorname{det}
\begin{bmatrix}
1&2&3&4\\
0&-1&1&3\\
1&6&-1&2\\
1&-1&1&0 
\end{bmatrix}
\end{aligned}
$$
We can see that performing a cofactor expansion about the 3rd row reduces the computation finding the determinant of just a single $4\times4$ matrix.  Now we can perform another cofactor expansion on this matrix.
$$
\operatorname{det}
\begin{bmatrix}
1&2&3&4\\
0&-1&1&3\\
1&6&-1&2\\
1&-1&1&0 
\end{bmatrix}
j = 1
$$

$$
\begin{aligned}
& = \operatorname{det}
\begin{bmatrix}
-1&1&3\\
6&-1&2\\
-1&1&0 
\end{bmatrix} - 0\operatorname{det}
\begin{bmatrix}
2&3&4\\
6&-1&2\\
-1&1&0 
\end{bmatrix} + \operatorname{det}
\begin{bmatrix}
2&3&4\\
-1&1&3\\
-1&1&0 
\end{bmatrix} - \operatorname{det}
\begin{bmatrix}
2&3&4\\
-1&1&3\\
6&-1&2\\
\end{bmatrix} \\
& = \operatorname{det}
\begin{bmatrix}
-1&1&3\\
6&-1&2\\
-1&1&0 
\end{bmatrix} + \operatorname{det}
\begin{bmatrix}
2&3&4\\
-1&1&3\\
-1&1&0 
\end{bmatrix} - \operatorname{det}
\begin{bmatrix}
2&3&4\\
-1&1&3\\
6&-1&2\\
\end{bmatrix} 
\end{aligned}
$$
Now we can find the determinant of these $3\times3$ matrices.
$$
\operatorname{det}
\begin{bmatrix}
-1&1&3\\
6&-1&2\\
-1&1&0 
\end{bmatrix} j = 3
$$
$$
\begin{aligned}
& = 3\operatorname{det}
\begin{bmatrix}
6&-1\\
-1&1 
\end{bmatrix} - 2\operatorname{det}
\begin{bmatrix}
-1&1\\
-1&1 
\end{bmatrix} + 0\operatorname{det}
\begin{bmatrix}
-1&1\\
6&-1\\
\end{bmatrix} \\
& = 3(5) - 2(0) \\
& = 15
\end{aligned}
$$
$$
\operatorname{det}
\begin{bmatrix}
2&3&4\\
-1&1&3\\
-1&1&0 
\end{bmatrix} j = 3
$$
$$
\begin{aligned}
& = 4\operatorname{det}
\begin{bmatrix}
-1&1\\
-1&1 
\end{bmatrix} -3
\operatorname{det}
\begin{bmatrix}
2&3\\
-1&1
\end{bmatrix} +
0\operatorname{det}
\begin{bmatrix}
2&3\\
-1&1\\
\end{bmatrix} \\
& = 4(0) - 3(5) \\
& = -15
\end{aligned}
$$
$$
\operatorname{det}
\begin{bmatrix}
2&3&4\\
-1&1&3\\
6&-1&2\\
\end{bmatrix} j=2
$$
$$
\begin{aligned}
& = -3\operatorname{det}
\begin{bmatrix}
-1&3\\
6&2\\
\end{bmatrix} +
\operatorname{det}
\begin{bmatrix}
2&4\\
6&2\\
\end{bmatrix} +
\operatorname{det}
\begin{bmatrix}
2&4\\
-1&3\\
\end{bmatrix} \\
& = -3(-20) + (-20) + (10) \\
& = 60 - 20 + 10 \\
& = 50
\end{aligned}
$$
Since we have calculated the determinants of all the minor matrices,  we can now find the sum to find the determinant of the original matrix,  and we find that
$$
\begin{aligned}
\operatorname{det}
\begin{bmatrix}
1&2&3&4&5\\
0&-1&1&3&2\\
0&0&0&0&1\\
1&6&-1&2&4\\
1&-1&1&0&1 
\end{bmatrix} & =
15 - 15 - 50 \\
& = -50
\end{aligned}
$$

\qs{}{
Compute the determinant of each of the following
$$
A = 
\begin{bmatrix}
1&47&\pi&\ln(2)&\sqrt{2}\\ 
0&2 & \sqrt{3}&100&-5\\
0&0&3&-\pi&2\\
0&0&0&4&47^2 \\
0&0&0&0&5 
\end{bmatrix}\qquad\qquad B = 
\begin{bmatrix} 
2&2&0&0&0&0&0&0\\
3&1&0&0&0&0&0&0\\
0&0&1&-1&3&0&0&0\\
0&0&-1&1&0&0&0&0\\
0&0&0&0&1&0&0&0\\
0&0&0&0&0&11&0&0\\
0&0&0&0&0&0&4&-1\\
0&0&0&0&0&0&1&2
\end{bmatrix}$$
}
\begin{note}
Recall:
\begin{itemize}
\item If a matrix A is either triangular or diagonal, the determinant can be computed by taking the product of its diagonal elements
$$
A = \begin{bmatrix}
a_{11} & a_{12} & \ldots & a_{1n} \\
& a_{22} & & a_{2n} \\
& & \ddots &  \vdots \\
& & & a_{nn} \end{bmatrix} \qquad
A = \begin{bmatrix}
a_{11} & & &  \\
a_{21} & a_{22} & & \\
\vdots & & \ddots & \\
a_{n1} & a_{n2} &  \ldots & a_{nn}
\end{bmatrix} \qquad
A = \begin{bmatrix}
a_{11} & & & \\
& a_{22} & & \\
& & \ddots & \\
& & & a_{nn}
\end{bmatrix}
$$
$$
\operatorname{det}(A) = a_{11} \cdot a_{22} \cdot \ldots \cdot a_{nn}
$$
\item Similarly,  if a matrix A is either block triangular or block diagonal, the determinant can be computed by taking the product of the determinants of its diagonal elements
$$
\begin{aligned}
A & = \begin{bmatrix}
A_{11} & A_{12} & \ldots & A_{1n} \\
& A_{22} & & A_{2n} \\
& & \ddots & \vdots \\
& & & A_{nn}
\end{bmatrix} \\
\operatorname{det}(A) & = \operatorname{det}(A_{11}) \cdot \operatorname{det}(A_{22}) \cdot \ldots \cdot \operatorname{det}(A_{nn})
\end{aligned}
$$
\end{itemize}
\end{note}
\sol Since $A$ is triangular,  we can find its determinant by taking the product of its diagonal elements 
$$
\operatorname{det}\begin{bmatrix}
1&47&\pi&\ln(2)&\sqrt{2}\\ 
0&2 & \sqrt{3}&100&-5\\
0&0&3&-\pi&2\\
0&0&0&4&47^2 \\
0&0&0&0&5 
\end{bmatrix} = 1 \cdot 2 \cdot 3 \cdot 4 \cdot 5 = 120
$$
Since $B$ is block triangular,  we can find its determinant by taking the product of the determinants of its diagonal elements.  Notice that I call this matrix block triangular instead of block diagonal. This is because the result will be the same however if we are to call it block diagonal then the largest matrix we need to compute the determinant of is $3\times3$, while if we call it block triangular then the largest matrix is $2\times2$
$$
\begin{aligned}
\operatorname{det}\begin{bmatrix} 
2&2&0&0&0&0&0&0\\
3&1&0&0&0&0&0&0\\
0&0&1&-1&3&0&0&0\\
0&0&-1&1&0&0&0&0\\
0&0&0&0&1&0&0&0\\
0&0&0&0&0&11&0&0\\
0&0&0&0&0&0&4&-1\\
0&0&0&0&0&0&1&2
\end{bmatrix} & = 
\operatorname{det}\begin{bmatrix}
2 & 2 \\
3 & 1
\end{bmatrix} \cdot \operatorname{det}\begin{bmatrix}
1 & -1\\
-1 & 1
\end{bmatrix} \cdot \operatorname{det}\left[1\right] \cdot
\left[11\right] \cdot \operatorname{det}\begin{bmatrix}
4 & 1 \\ 1 & 2
\end{bmatrix} \\
& = (2 - 6) \cdot (1 - 1) \cdot 1 \cdot 11 \cdot (8 - 1) \\
& = -4 \cdot 0 \cdot 11 \cdot 7 \\
& = 0
\end{aligned}
$$
\qs{}{
(a)Compute the determinant of the following $16\times 16$ matrix $E$:
$$\setcounter{MaxMatrixCols}{20}E=\begin{bmatrix}
1&0&0&0&0&0&0&0&        1&1&1&0&1&-1&1&0\\
0&-1&0&0&0&0&0&0&       2&-1&0&1&-1&2&-1&1\\
0&0&1&0&0&0&0&0&        3&1&1&0&2&0&2&1\\
0&0&0&2&0&0&0&0&        0&-2&2&0&0&2&0&2\\
0&0&0&0&1&0&0&0&        1&1&0&1&-1&2&1&0\\
0&0&0&0&0&1/2 &0&0&        0&1&1&0&1&0&1&0\\
0&0&0&0&0&0&1&0&        -1&1&1&1&0&1&-1&-1\\
0&0&0&0&0&0&0&-1&       2&-1&2&1&0&2&-3&0\\

1&0&0&0&0&0&0&0&        2&1&1&0&1&-1&1&0\\
0&-1&0&0&0&0&0&0&       2&0&0&1&-1&2&-1&1\\
0&0&2&0&0&0&0&0&        6&2&3&0&4&0&4&2\\
0&0&0&3&0&0&0&0&        0&-3&3&1&0&3&0&3\\
0&0&0&0&1&0&0&0&        1&1&0&1&-2&2&1&0\\
0&0&0&0&0&-1&0&0&       0&-2&-2&0&-2&1&-2&0\\
0&0&0&0&0&0&1&0&        -1&1&1&1&0&1&-2&-1\\
0&0&0&0&0&0&0&2&        -4&2&-4&-2&0&-4&6&1
\end{bmatrix}$$ \\
(b) Do the columns of the above matrix form a basis for $\mathbb{R}^{16}$? What about the rows, do the rows of $E$ form a basis for $\mathbb{R}^{16}$?\\
(c) What is $\operatorname{det}(E^\top)$?
}
\begin{note}
Recall:
\begin{itemize}
\item If $E$ is a blocked off matrix
$$
E = \begin{bmatrix}
A & B \\ C & D
\end{bmatrix}
$$
and $A$ is an invertible matrix,  then $\det(E)=\det(A)\det(D-CA^{-1}B)$
\item If a matrix $A\in\mathbb{R}^{n\times n}$ is a diagonal matrix,  then the inverse can be computed by taking the reciprocal of each diagonal element
$$
A^{-1} = \begin{bmatrix}
1/a_{11} & & & \\
& 1/a_{22} & & \\
& & \ddots & \\
& & & 1/a_{nn}
\end{bmatrix}
$$
\item $\det(A) = \det(A^{\top})$
\end{itemize}
\end{note}
\sol \\
(a) We can divide this matrix into 4 unique $8\times8$ matrices,
$$
A = \begin{bmatrix}
1&0&0&0&0&0&0&0\\
0&-1&0&0&0&0&0&0\\
0&0&1&0&0&0&0&0\\
0&0&0&2&0&0&0&0\\
0&0&0&0&1&0&0&0\\
0&0&0&0&0&1/2 &0&0\\
0&0&0&0&0&0&1&0\\
0&0&0&0&0&0&0&-1
\end{bmatrix} \qquad \qquad
B = \begin{bmatrix}
1&1&1&0&1&-1&1&0\\
2&-1&0&1&-1&2&-1&1\\
3&1&1&0&2&0&2&1\\
0&-2&2&0&0&2&0&2\\
1&1&0&1&-1&2&1&0\\
0&1&1&0&1&0&1&0\\
-1&1&1&1&0&1&-1&-1\\
2&-1&2&1&0&2&-3&0\\
\end{bmatrix}
$$
$$
C =
\begin{bmatrix}
1&0&0&0&0&0&0&0 \\
0&-1&0&0&0&0&0&0 \\
0&0&2&0&0&0&0&0 \\
0&0&0&3&0&0&0&0 \\
0&0&0&0&1&0&0&0 \\
0&0&0&0&0&-1&0&0 \\
0&0&0&0&0&0&1&0 \\
0&0&0&0&0&0&0&2
\end{bmatrix} \qquad \qquad
D = \begin{bmatrix}
2&1&1&0&1&-1&1&0\\
2&0&0&1&-1&2&-1&1\\
6&2&3&0&4&0&4&2\\
0&-3&3&1&0&3&0&3\\
1&1&0&1&-2&2&1&0\\
0&-2&-2&0&-2&1&-2&0\\
-1&1&1&1&0&1&-2&-1\\
-4&2&-4&-2&0&-4&6&1
\end{bmatrix}
$$
As seen previously,  we can find the determinant of a diagonal matrix by finding the product of the diagonal elements,  so we can quickly find the $\det(A)$
$$
\begin{aligned}
\det(A) & = 1 \cdot -1 \cdot 1 \cdot 2 \cdot 1 \cdot 1/2 \cdot 1 \cdot -1 \\
& = 1
\end{aligned}
$$
Now we must find the inverse of $A$ $(A^{-1})$ This is easy since $A$ is a diagonal matrix
$$
A^{-1} = \begin{bmatrix}
1&0&0&0&0&0&0&0\\
0&-1&0&0&0&0&0&0\\
0&0&1&0&0&0&0&0\\
0&0&0&1/2&0&0&0&0\\
0&0&0&0&1&0&0&0\\
0&0&0&0&0&2&0&0\\
0&0&0&0&0&0&1&0\\
0&0&0&0&0&0&0&-1
\end{bmatrix}
$$
Now we can compute the operations to find the matrix that results from $D - CA^{-1}B$,  Firstly lets find $CA^{-1}B.$
$$
\begin{aligned}
CA^{-1} & = 
\begin{bmatrix}
1&0&0&0&0&0&0&0 \\
0&-1&0&0&0&0&0&0 \\
0&0&2&0&0&0&0&0 \\
0&0&0&3&0&0&0&0 \\
0&0&0&0&1&0&0&0 \\
0&0&0&0&0&-1&0&0 \\
0&0&0&0&0&0&1&0 \\
0&0&0&0&0&0&0&2
\end{bmatrix}
\begin{bmatrix}
1&0&0&0&0&0&0&0\\
0&-1&0&0&0&0&0&0\\
0&0&1&0&0&0&0&0\\
0&0&0&1/2&0&0&0&0\\
0&0&0&0&1&0&0&0\\
0&0&0&0&0&2&0&0\\
0&0&0&0&0&0&1&0\\
0&0&0&0&0&0&0&-1
\end{bmatrix} \\
& = \begin{bmatrix}
1&0&0&0&0&0&0&0\\
0&1&0&0&0&0&0&0\\
0&0&2&0&0&0&0&0\\
0&0&0&3/2&0&0&0&0\\
0&0&0&0&1&0&0&0\\
0&0&0&0&0&-2&0&0\\
0&0&0&0&0&0&1&0\\
0&0&0&0&0&0&0&-2
\end{bmatrix}
\end{aligned}
$$
With the matrix $CA^{-1}$,  we can multiply by $B$ to find $CA^{-1}B$
$$
\begin{aligned}
CA^{-1}B & = \begin{bmatrix}
1&0&0&0&0&0&0&0\\
0&1&0&0&0&0&0&0\\
0&0&2&0&0&0&0&0\\
0&0&0&3/2&0&0&0&0\\
0&0&0&0&1&0&0&0\\
0&0&0&0&0&-2&0&0\\
0&0&0&0&0&0&1&0\\
0&0&0&0&0&0&0&-2
\end{bmatrix}
\begin{bmatrix}
1&1&1&0&1&-1&1&0\\
2&-1&0&1&-1&2&-1&1\\
3&1&1&0&2&0&2&1\\
0&-2&2&0&0&2&0&2\\
1&1&0&1&-1&2&1&0\\
0&1&1&0&1&0&1&0\\
-1&1&1&1&0&1&-1&-1\\
2&-1&2&1&0&2&-3&0\\
\end{bmatrix} \\
& =
\begin{bmatrix}
1&1&1&0&1&-1&1&0\\
2&-1&0&1&-1&2&-1&1\\
6&2&2&0&4&0&4&2\\
0&-3&3&0&0&3&0&3\\
1&1&0&1&-1&2&1&0\\
0&-2&-2&0&-2&0&-2&0\\
-1&1&1&1&0&1&-1&-1\\
-4&2&-4&-2&0&-4&6&0
\end{bmatrix}
\end{aligned}
$$
Now we can finally compute $D - CA^{-1}B$, 
$$
\begin{aligned}
D - CA^{-1}B & = 
\begin{bmatrix}
2&1&1&0&1&-1&1&0\\
2&0&0&1&-1&2&-1&1\\
6&2&3&0&4&0&4&2\\
0&-3&3&1&0&3&0&3\\
1&1&0&1&-2&2&1&0\\
0&-2&-2&0&-2&1&-2&0\\
-1&1&1&1&0&1&-2&-1\\
-4&2&-4&-2&0&-4&6&1
\end{bmatrix} -
\begin{bmatrix}
1&1&1&0&1&-1&1&0\\
2&-1&0&1&-1&2&-1&1\\
6&2&2&0&4&0&4&2\\
0&-3&3&0&0&3&0&3\\
1&1&0&1&-1&2&1&0\\
0&-2&-2&0&-2&0&-2&0\\
-1&1&1&1&0&1&-1&-1\\
-4&2&-4&-2&0&-4&6&0
\end{bmatrix} \\
& = \begin{bmatrix}
1&0&0&0&0&0&0&0\\
0&1&0&0&0&0&0&0\\
0&0&1&0&0&0&0&0\\
0&0&0&1&0&0&0&0\\
0&0&0&0&-1&0&0&0\\
0&0&0&0&0&1&0&0\\
0&0&0&0&0&0&-1&0\\
0&0&0&0&0&0&0&1
\end{bmatrix}
\end{aligned}
$$
Since we have found the matrix $D - CA^{-1}B$,  and have also found it to be a diagonal matrix,  the determinant will be the product of the diagonal elements
$$
\begin{aligned}
\det(D - CA^{-1}B) & = 1 \cdot 1 \cdot 1 \cdot 1 \cdot -1 \cdot 1 \cdot -1 \cdot 1 \\
& = 1
\end{aligned}
$$
So now we can compute the $\det(E)$ by finding the product of the two determinants we have found,  in particular
$$
\begin{aligned}
\det(E) & = \det(A)\det(D-CA^{-1}B) \\
& = (1)(1) \\
& = 1
\end{aligned}
$$
(b) The result of the determinant calculated in part (a) reveals all the information necessary to answer both of thes questions.  We know from the Invertible Matrix Theorem a few statements are equivalent for a given a matrix $A\in\mathbb{R}^{n\times n}$.  In particular we see that
\begin{itemize}
\item $\det(A) \neq 0$
\item The columns of $A$ are linearly independent
\item The columns of $A$ span $\mathbb{R}^n$
\item The rows of $A$ are linearly independent.
\item The rows of $A$ span $\mathbb{R}^n$
\end{itemize}
We computed the $\det(E)$ to be 1,  meaning that the matrix is invertible though that is not important or information that is necessary to answer this question.  By the Invertible Matrix Theorem,  we know that the columns of $E$ are linearly independent,  as well as spanning of $\mathbb{R}^{16}$,  so thus the columns form a basis for $\mathbb{R}^{16}$. \\
\\ \noindent Further, the Invertible Matrix Theorem also tells us that since $\det(E)$ is nonzero then the rows are linearly independent and also spanning of $\mathbb{R}^{16}$,  and thus the rows also form a basis for $\mathbb{R}^{16}$.  \\
\\ \noindent(c) $\det(E^{\top}) = \det(E) = 1$
\qs{}{
(a) Find the characteristic polynomial of $B$ from problem 2. \\
(b) Find all eigenvalues of $B$ \\
(c) For each eigenvalue of $B$, find an eigenvector.
}
\begin{note}
Recall:
\begin{itemize}
\item The characteristic polynomial of $A\in\mathbb{R}^{n\times n}$ is $\det(\lambda I_n - A)$
\item $\vec{0} = \lambda I_n \vec{x} - A\vec{x} = (\lambda I_n - A)\vec{x}$
\end{itemize}
\end{note}
\sol \\
(a) Characteristic polynomial of $B$
$$
\begin{aligned}
\det(\lambda I_8 - B) = \det\begin{bmatrix}
\lambda-2&-2&0&0&0&0&0&0\\
-3&\lambda-1&0&0&0&0&0&0\\
0&0&\lambda-1&1&-3&0&0&0\\
0&0&1&\lambda-1&0&0&0&0 \\
0&0&0&0&\lambda-1&0&0&0\\
0&0&0&0&0&\lambda-11&0&0\\
0&0&0&0&0&0&\lambda-4&1\\
0&0&0&0&0&0&-1&\lambda-2
\end{bmatrix}
\end{aligned}
$$
We can use the same principle as before to find the determinant of a block triangular matrix,  so that we can find the eigenvalues
\begin{align*}
& = ((\lambda-2)(\lambda-1)-6)((\lambda-1)(\lambda-1)-1)(\lambda-1)(\lambda-11)((\lambda-4)(\lambda-2)+1) \\
& = (\lambda^2 -3\lambda -4)(\lambda^2-2\lambda)(\lambda-1)(\lambda-11)(\lambda^2-6\lambda+9) \\
& = (\lambda-4)(\lambda+1)(\lambda)(\lambda-2)(\lambda-1)(\lambda-11)(\lambda-3)^2
\end{align*}
(b) With the characteristic polynomial of $B$ found from part (a),  We can find the zeros to get the eigenvalues of $B$.  We find that
\begin{itemize}
\begin{minipage}[t]{0.5\textwidth}
\item $\lambda_1=4$
\item $\lambda_2=-1$
\item $\lambda_3=0$
\item $\lambda_4=2$
\end{minipage}
\begin{minipage}[t]{0.5\textwidth}
\item $\lambda_5=1$
\item $\lambda_6=11$
\item $\lambda_7=3$
\end{minipage}
\end{itemize}
(c) To solve for the eigenvectors of the matrix $B$,  We must find the vector that when multiplied by the characteristic matrix results in the zero vector
\begin{itemize}
\item $\lambda = 4$
$$
\begin{bmatrix}
2&-2&0&0&0&0&0&0\\
-3&3&0&0&0&0&0&0\\
0&0&3&1&-3&0&0&0\\
0&0&1&3&0&0&0&0\\
0&0&0&0&3&0&0&0\\
0&0&0&0&0&-7&0&0\\
0&0&0&0&0&0&0&1\\
0&0&0&0&0&0&-1&2
\end{bmatrix}
\begin{bmatrix}
x_1 \\ x_2 \\ x_3 \\ x_4 \\ x_5 \\ x_6 \\ x_7 \\ x_8
\end{bmatrix} = \begin{bmatrix}
0 \\ 0 \\ 0 \\ 0 \\ 0 \\ 0 \\ 0 \\ 0
\end{bmatrix}
$$
We can construct a system of equations to find the values of each $x_i$.
$$
\left\{
\begin{aligned}
2x_1 - 2x_2 & = 0 \\
3x_3+x_4-3x_5 & = 0 \\
x_3 + x_4 & = 0 \\
3x_5 & = 0 \\
-7x_6 & = 0 \\
x_8 & = 0 \\
-x_7 + 2x_8 & = 0
\end{aligned}
\right.
$$
Immediately,  we can see that $x_5=x_6=x_7=x_8=0$.  What is not immediately evident,  that we can manipulate the third equation to see that $x_4 = -x_3$ and then plug that into the second equation to see that $3x_3 - x_3 = 0$.  We see from the first equation that $x_1 = x_2$,  and for the purpose of this calculation,  lets say $x_2 = a$,  for some $a\in\mathbb{R}$.  We can now determine the eigenvector when $\lambda = 4$,  Since nonzero scalar multiples of an eigenvector are equivalent to the original eigenvector for a given eigenvalue. 
$$
\begin{bmatrix}
a \\ a \\ 0 \\ 0 \\ 0 \\ 0 \\ 0 \\ 0
\end{bmatrix} =
a\begin{bmatrix}
1 \\ 1 \\ 0 \\ 0 \\ 0 \\ 0 \\ 0 \\ 0
\end{bmatrix} \implies
\vec{x} = \begin{bmatrix}
1 \\ 1 \\ 0 \\ 0 \\ 0 \\ 0 \\ 0 \\ 0
\end{bmatrix}
$$
\item $\lambda = -1$
$$
\begin{bmatrix}
-3&-2&0&0&0&0&0&0\\
-3&-2&0&0&0&0&0&0\\
0&0&-2&1&-3&0&0&0\\
0&0&1&-2&0&0&0&0\\
0&0&0&0&-2&0&0&0\\
0&0&0&0&0&-12&0&0\\
0&0&0&0&0&0&-5&1\\
0&0&0&0&0&0&-1&-3
\end{bmatrix}
\begin{bmatrix}
x_1 \\ x_2 \\ x_3 \\ x_4 \\ x_5 \\ x_6 \\ x_7 \\ x_8
\end{bmatrix} = \begin{bmatrix}
0 \\ 0 \\ 0 \\ 0 \\ 0 \\ 0 \\ 0 \\ 0
\end{bmatrix}
$$
We can construct a system of equations to find the values of each $x_i$.
$$
\left\{
\begin{aligned}
-3x_1 - 2x_2 & = 0 \\
-2x_3+x_4-3x_5 & = 0 \\
x_3 - 2x_4 & = 0 \\
-2x_5 & = 0 \\
-12x_6 & = 0 \\
-3x_7 + x_8 & = 0 \\
-x_7 - 3x_8 & = 0
\end{aligned}
\right.
$$
We can see that $x_5 = x_6 = 0$ very clearly from this system.  We can also determine that $x_7 = x_ 8 = 0$ from the final two equations,  because if we manipulate them around a bit,  we can arrive at $10x_8 = 0$ which implies that $x_7 = 0$.  A similar manipulation can be done on equations 2 and 3 to arrive at $3x_4 = 0$,  which suggests that $x_4 = x_3 = 0$.  We are left with $-3x_1 = 2x_2$.  We can say that $x_1 = -3/2x_2$,  and if we say $x_2 = a$ where $a\in\mathbb{R}$,  we can solve for our original eigenvector,  in particular we see that
$$
\begin{bmatrix}
-3/2a \\ a \\ 0 \\ 0 \\ 0 \\ 0 \\ 0 \\ 0
\end{bmatrix} =
a\begin{bmatrix}
-3/2 \\ 1 \\ 0 \\ 0 \\ 0 \\ 0 \\ 0 \\ 0
\end{bmatrix} \implies
\vec{x} = \begin{bmatrix}
-3/2 \\ 1 \\ 0 \\ 0 \\ 0 \\ 0 \\ 0 \\ 0
\end{bmatrix}
$$
\item $\lambda = 0$
$$
\begin{bmatrix}
-2&-2&0&0&0&0&0&0\\
-3&-1&0&0&0&0&0&0\\
0&0&-1&1&-3&0&0&0\\
0&0&1&-1&0&0&0&0\\
0&0&0&0&-1&0&0&0\\
0&0&0&0&0&-11&0&0\\
0&0&0&0&0&0&-4&1\\
0&0&0&0&0&0&-1&-2
\end{bmatrix}
\begin{bmatrix}
x_1 \\ x_2 \\ x_3 \\ x_4 \\ x_5 \\ x_6 \\ x_7 \\ x_8
\end{bmatrix} = \begin{bmatrix}
0 \\ 0 \\ 0 \\ 0 \\ 0 \\ 0 \\ 0 \\ 0
\end{bmatrix}
$$
We can construct a system of equations to find the values of each $x_i$.
$$
\left\{
\begin{aligned}
-2x_1 - 2x_2 & = 0 \\
-3x_1 - x_2 & = 0 \\
-x_3+x_4-3x_5 & = 0 \\
x_3 - x_4 & = 0 \\
-x_5 & = 0 \\
-11x_6 & = 0 \\
-4x_7 + x_8 & = 0 \\
-x_7 - 2x_8 & = 0
\end{aligned}
\right.
$$
From just looking at the system,  we can see that $x_1 = x_2 = x_ 5 = x_6 = x_7 = x_8 = 0$.  We are tasked now with determining if $x_3$ and $x_4$ are nonzero.  Equation 4 tells us that $x_3 = x_4$,  and if we are to say $x_4 = a$ where $a\in\mathbb{R}$,  then we can solve for the original eigenvector,  in particular we see that
$$
\begin{bmatrix}
0 \\ 0 \\ a \\ a \\ 0 \\ 0 \\ 0 \\ 0
\end{bmatrix} =
a\begin{bmatrix}
0 \\ 0 \\ 1 \\ 1 \\ 0 \\ 0 \\ 0 \\ 0
\end{bmatrix} \implies
\vec{x} = \begin{bmatrix}
0 \\ 0 \\ 1 \\ 1 \\ 0 \\ 0 \\ 0 \\ 0
\end{bmatrix}
$$ 
\item $\lambda = 2$
$$
\begin{bmatrix}
0&-2&0&0&0&0&0&0\\
-3&1&0&0&0&0&0&0\\
0&0&1&1&-3&0&0&0\\
0&0&1&1&0&0&0&0\\
0&0&0&0&1&0&0&0\\
0&0&0&0&0&-9&0&0\\
0&0&0&0&0&0&-2&1\\
0&0&0&0&0&0&-1&0
\end{bmatrix}
\begin{bmatrix}
x_1 \\ x_2 \\ x_3 \\ x_4 \\ x_5 \\ x_6 \\ x_7 \\ x_8
\end{bmatrix} = \begin{bmatrix}
0 \\ 0 \\ 0 \\ 0 \\ 0 \\ 0 \\ 0 \\ 0
\end{bmatrix}
$$
We can construct a system of equations to find the values of each $x_i$.
$$
\left\{
\begin{aligned}
-2x_2 & = 0 \\
-3x_1 + x_2 & = 0 \\
x_3+x_4-3x_5 & = 0 \\
x_3 + x_4 & = 0 \\
x_5 & = 0 \\
-9x_6 & = 0 \\
-2x_7 + x_8 & = 0 \\
-x_7 & = 0
\end{aligned}
\right.
$$
From the system of equations,  we find that $x_1 = x_2 = x_5 = x_6 = x_7 = x_8 = 0$. This means that we are left needing to find the values of $x_3$ and $x_4$.  From equation 4,  we can see that $x_3 = -x_4$.  If we say $x_4=a$,  where $a\in\mathbb{R}$,  then we have all the information we need to solve for the original eigenvector when $\lambda = 2$.  In particular,  we see that
$$
\begin{bmatrix}
0 \\ 0 \\ -a \\ a \\ 0 \\ 0 \\ 0 \\ 0
\end{bmatrix} =
a\begin{bmatrix}
0 \\ 0 \\ -1 \\ 1 \\ 0 \\ 0 \\ 0 \\ 0
\end{bmatrix} \implies
\vec{x} = \begin{bmatrix}
0 \\ 0 \\ -1 \\ 1 \\ 0 \\ 0 \\ 0 \\ 0
\end{bmatrix}
$$ 
\item $\lambda = 1$
$$
\begin{bmatrix}
-1&-2&0&0&0&0&0&0\\
-3&0&0&0&0&0&0&0\\
0&0&0&1&-3&0&0&0\\
0&0&1&0&0&0&0&0\\
0&0&0&0&0&0&0&0\\
0&0&0&0&0&-10&0&0\\
0&0&0&0&0&0&-3&1\\
0&0&0&0&0&0&-1&-1
\end{bmatrix}
\begin{bmatrix}
x_1 \\ x_2 \\ x_3 \\ x_4 \\ x_5 \\ x_6 \\ x_7 \\ x_8
\end{bmatrix} = \begin{bmatrix}
0 \\ 0 \\ 0 \\ 0 \\ 0 \\ 0 \\ 0 \\ 0
\end{bmatrix}
$$
We can construct a system of equations to find the values of each $x_i$.
$$
\left\{
\begin{aligned}
-x_1 - 2x_2 & = 0 \\
-3x_1 + x_2 & = 0 \\
x_4-3x_5 & = 0 \\
x_3 & = 0 \\
-10x_6 & = 0 \\
-3x_7 + x_8 & = 0 \\
-x_7 - x_8 & = 0
\end{aligned}
\right.
$$
The system of equations reveals that $x_1 = x_2 = x_3 = x_6 = x_7 = x_8 = 0$.  We can use the third equation in the system to determine the values of $x_4$ and $x_5$.  From the third equation,  we can arrive at $x_4 = 3x_5$,  and if we say that $x_5 = a$ where $a\in\mathbb{R}$ then we have all the information we need to be able to determine the original eigenvector when $\lambda = 1$.  In particular we see that
$$
\begin{bmatrix}
0 \\ 0 \\ 0 \\ 3a \\ a \\ 0 \\ 0 \\ 0
\end{bmatrix} =
a\begin{bmatrix}
0 \\ 0 \\ 0 \\ 3 \\ 1 \\ 0 \\ 0 \\ 0
\end{bmatrix} \implies
\vec{x} = \begin{bmatrix}
0 \\ 0 \\ 0 \\ 3 \\ 1 \\ 0 \\ 0 \\ 0
\end{bmatrix}
$$ 
\item $\lambda = 11$
$$
\begin{bmatrix}
9&-2&0&0&0&0&0&0\\
-3&10&0&0&0&0&0&0\\
0&0&10&1&-3&0&0&0\\
0&0&1&10&0&0&0&0\\
0&0&0&0&10&0&0&0\\
0&0&0&0&0&0&0&0\\
0&0&0&0&0&0&7&1\\
0&0&0&0&0&0&-1&9
\end{bmatrix}
\begin{bmatrix}
x_1 \\ x_2 \\ x_3 \\ x_4 \\ x_5 \\ x_6 \\ x_7 \\ x_8
\end{bmatrix} = \begin{bmatrix}
0 \\ 0 \\ 0 \\ 0 \\ 0 \\ 0 \\ 0 \\ 0
\end{bmatrix}
$$
We can construct a system of equations to find the values of each $x_i$.
$$
\left\{
\begin{aligned}
9x_1 - 2x_2 & = 0 \\
-3x_1 + 10x_2 & = 0 \\
10x_3 + x_4-3x_5 & = 0 \\
x_3 + 10x_4 & = 0 \\
7x_7 + x_8 & = 0 \\
-x_7 + 9x_8 & = 0
\end{aligned}
\right.
$$
From the system,  we can see that $x_1 = x_2 = x_3 = x_4 = x_ 5 = x_7 = x_8 = 0$.  The system does not account for the value of $x_6$,  so for the purpose of finding the eigenvector,  we can call $x_6 = a$,  where $a\in\mathbb{R}$.  With this,  we can solve for the original eigenvector and in particular we see that
$$
\begin{bmatrix}
0 \\ 0 \\ 0 \\ 0 \\ 0 \\ a \\ 0 \\ 0
\end{bmatrix} =
a\begin{bmatrix}
0 \\ 0 \\ 0 \\ 0 \\ 0 \\ 1 \\ 0 \\ 0
\end{bmatrix} \implies
\vec{x} = \begin{bmatrix}
0 \\ 0 \\ 0 \\ 0 \\ 0 \\ 1 \\ 0 \\ 0
\end{bmatrix}
$$ 
\item $\lambda = 3$
$$
\begin{bmatrix}
1&-2&0&0&0&0&0&0\\
-3&2&0&0&0&0&0&0\\
0&0&2&1&-3&0&0&0\\
0&0&1&2&0&0&0&0\\
0&0&0&0&2&0&0&0\\
0&0&0&0&0&-8&0&0\\
0&0&0&0&0&0&-1&1\\
0&0&0&0&0&0&-1&1
\end{bmatrix}
\begin{bmatrix}
x_1 \\ x_2 \\ x_3 \\ x_4 \\ x_5 \\ x_6 \\ x_7 \\ x_8
\end{bmatrix} = \begin{bmatrix}
0 \\ 0 \\ 0 \\ 0 \\ 0 \\ 0 \\ 0 \\ 0
\end{bmatrix}
$$
We can construct a system of equations to find the values of each $x_i$.
$$
\left\{
\begin{aligned}
x_1 - 2x_2 & = 0 \\
-3x_1 + 2x_2 & = 0 \\
2x_3 + x_4-3x_5 & = 0 \\
x_3 + 2x_4 & = 0 \\
-8x_6 & = 0 \\
-x_7 + x_8 & = 0
\end{aligned}
\right.
$$
We can see from the system that $x_1 = x_2 = x_3 = x_4 = x_5 = x_6 = 0$.  The final equation reveals that $x_7 = x_8$.  If we take $x_8=a$ where $a\in\mathbb{R}$,  then we have the sufficient information to find the original eigenvector when $\lambda = 3$.  In particular we see that
$$
\begin{bmatrix}
0 \\ 0 \\ 0 \\ 0 \\ 0 \\ 0 \\ a \\ a
\end{bmatrix} =
a\begin{bmatrix}
0 \\ 0 \\ 0 \\ 0 \\ 0 \\ 0 \\ 1 \\ 1
\end{bmatrix} \implies
\vec{x} = \begin{bmatrix}
0 \\ 0 \\ 0 \\ 0 \\ 0 \\ 0 \\ 1 \\ 1
\end{bmatrix}
$$ 
\end{itemize}
\qs{}{
Find the inverse and determinant of the following matrix:
$$\begin{bmatrix} 
2&2&0&0&0&0&0&0\\
3&1&0&0&0&0&0&0\\
0&0&2&1&3&0&0&0\\
0&0&3&2&0&0&0&0\\
0&0&0&0&1&0&0&0\\
0&0&0&0&0&11&0&0\\
0&0&0&0&0&0&4&7\\
0&0&0&0&0&0&1&2
\end{bmatrix}$$
}
\begin{note}
If a matrix $A$ is invertible and block diagonal,  then the inverse can be computed by finding the inverse of its diagonal elements,  in particular we see that
$$
A^{-1} =
\begin{bmatrix}
A_1^{-1} & & & \\
& A_2^{-1} & & \\
& & \ddots & \\
& & & A_n^{-1}
\end{bmatrix}
$$
\end{note}
To find the inverse,  we must find the inverse of the $3\times3$ block contained in the matrix.
\begin{align*}
&\begin{array}{ccc|ccc}
2&1&3&1&0&0\\
3&2&0&0&1&0\\
0&0&1&0&0&1
\end{array}
&& R_1 - R_2 \rightarrow R_1
&\begin{array}{ccc|ccc}
-1&-1&3&1&-1&0\\
3&2&0&0&1&0\\
0&0&1&0&0&1
\end{array}
&& R_1 - 3R_3 \rightarrow R_1\\
&\begin{array}{ccc|ccc}
-1&-1&0&1&-1&-3\\
3&2&0&0&1&0\\
0&0&1&0&0&1
\end{array}
&& R_2 + 3R_1 \rightarrow R_2
&\begin{array}{ccc|ccc}
-1&-1&0&1&-1&-3\\
0&-1&0&3&-2&-9\\
0&0&1&0&0&1
\end{array}
&& \begin{aligned} &-R_1 \rightarrow R_1 \\ &-R_2 \rightarrow R_2 \end{aligned} \\
&\begin{array}{ccc|ccc}
1&1&0&-1&1&3\\
0&1&0&-3&2&9\\
0&0&1&0&0&1
\end{array}
&& R_1 - R_2 \rightarrow R_1
&\begin{array}{ccc|ccc}
1&0&0&2&-1&-6\\
0&1&0&-3&2&9\\
0&0&1&0&0&1
\end{array}
\end{align*}
We can now commence with finding the inverse of the matrix
$$
\begin{aligned}
\begin{bmatrix} 
2&2&0&0&0&0&0&0\\
3&1&0&0&0&0&0&0\\
0&0&2&1&3&0&0&0\\
0&0&3&2&0&0&0&0\\
0&0&0&0&1&0&0&0\\
0&0&0&0&0&11&0&0\\
0&0&0&0&0&0&4&7\\
0&0&0&0&0&0&1&2
\end{bmatrix}^{-1} & = 
\begin{bmatrix}
\begin{bmatrix} 2 & 2 \\ 3 & 1 \end{bmatrix}^{-1} & & & \\
& \begin{bmatrix} 2 & 1 & 3 \\ 3 & 2 & 0 \\ 0 & 0 & 1 \end{bmatrix}^{-1} & & \\
& & \left[11\right]^{-1} & \\
& & & \begin{bmatrix} 4 & 7 \\ 1 & 2 \end{bmatrix}^{-1}
\end{bmatrix} \\
& = \begin{bmatrix}
-1/4\begin{bmatrix} 1 & -2 \\ -3 & 2 \end{bmatrix} & & & \\
& \begin{bmatrix} 2 & -1 & -6 \\ -3 & 2 & 9 \\ 0 & 0 & 1 \end{bmatrix} & & \\
& & 1/11 & \\
& & & \begin{bmatrix} 2 & -7 \\ -1 & 4 \end{bmatrix}
\end{bmatrix} \\
& = \begin{bmatrix} 
-1/4&1/2&0&0&0&0&0&0\\
3/4&-1/2&0&0&0&0&0&0\\
0&0&2&-1&-6&0&0&0\\
0&0&-3&2&9&0&0&0\\
0&0&0&0&1&0&0&0\\
0&0&0&0&0&1/11&0&0\\
0&0&0&0&0&0&2&-7\\
0&0&0&0&0&0&-1&4
\end{bmatrix}
\end{aligned}
$$
To find the determinant of the matrix we can apply the same principle we used earlier for finding the determinant of block triangular matrices.  That is to find the determinant of the diagonal blocks and find the product of these determinants
$$
\begin{aligned}
\det\begin{bmatrix} 
2&2&0&0&0&0&0&0\\
3&1&0&0&0&0&0&0\\
0&0&2&1&3&0&0&0\\
0&0&3&2&0&0&0&0\\
0&0&0&0&1&0&0&0\\
0&0&0&0&0&11&0&0\\
0&0&0&0&0&0&4&7\\
0&0&0&0&0&0&1&2
\end{bmatrix} &= \det\begin{bmatrix} 2 & 2 \\ 3 & 1 \end{bmatrix}\det\begin{bmatrix}2&1 \\3&2\end{bmatrix}\det[1]\det[11]\det\begin{bmatrix}4&7\\1&2\end{bmatrix} \\
& = -4 \cdot 1 \cdot 1 \cdot 11 \cdot 1 \\
& = -44
\end{aligned}
$$
\end{document}
